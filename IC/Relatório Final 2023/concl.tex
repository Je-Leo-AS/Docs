A evolução dos sistemas de comunicação sem fio tem promovido a implementação de diversos serviços móveis, tornando essencial que esses sistemas operem com máxima eficiência. Nesse cenário, a implementação de um DPD em cascata com o PA surge como uma alternativa de baixo custo e interessante para melhorar o desempenho desses sistemas.
O objetivo deste trabalho de iniciação cientifica foi a implementação e validação de um código em linguagem de descrição de hardware capaz de processar em tempo real as características de transferência de um amplificador. O código deve ser capaz de reproduzir não linearidades e efeitos de memória. Além disso, o código deve ocupar a quantidade mínima de recursos lógicos do circuito digital que irá sintetizá-lo, buscando reduzir o consumo de energia. Para isso, o projeto foi dividido em três etapas: estudo do DPD e da modelagem matemática, modelagem do DPD em software e implementação do DPD em hardware.
Sendo assim a primeira etapa de desenvolvimento do projeto foi a modelagem do PA em vírgula flutuante, utilizando o método do MP, para fazer essa modelagem utilizou-se um polinômio de 2° grau com uma amostra de memória, para fazer a validação dessa modelagem uitlizou-se a métrica do NMSE. Nesta etapa obteve-se um NMSE de -23,57 dB. A próxima etapa consiste em otimizar a quantidade de células lógicas utilizadas no processo limitando o número de bits utilizados. Nesta etapa, constatou-se que a partir de 8 bits não houve uma melhora significativa no NMSE. Dessa forma, essa foi a resolução em bits utilizada para a amostragem dos sinais. E por fim foi feito a modelagem do DPD em software o qual apresentou um comportamento inverso em relação o do PA, assim satisfazendo as necessidades. Iniciou-se a etapa de implementação do modelo em hardware. As próximas etapas incluem a validação e otimização do circuito, com o objetivo de alcançar a melhor e mais eficiente performance.