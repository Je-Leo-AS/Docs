A evolução dos sistemas de comunicação sem fio tem promovido a implementação de diversos serviços móveis, tornando essencial que esses sistemas operem com máxima eficiência. Nesse cenário, a implementação de um DPD em cascata com o PA surge como uma alternativa de baixo custo e interessante para melhorar o desempenho desses sistemas.
O objetivo deste trabalho de conclusão de curso é implementar em hardware um DPD baseado no modelo de Polinômio de Memória. Para isso, o projeto foi dividido em quatro etapas: estudo do DPD e da modelagem matemática, modelagem do DPD em software, implementação do DPD em hardware e, finalmente, design do circuito integrado.
Sendo assim a primeira etapa de desenvolvimento do projeto foi a modelagem do PA em vírgula flutuante, utilizando o método do MP, para fazer essa modelagem utilizou-se um polinômio de $2^\circ$ grau com uma amostra de memória, para fazer a validação dessa modelagem utilizou-se a métrica do NMSE. Nesta etapa obteve-se um NMSE de -23,57 dB, a próxima etapa consiste em otimizar a quantidade de células lógicas utilizadas no processo limitando o número de bits utilizados. Nesta etapa observou-se que a partir de 8 bits, não havia melhora expressiva no NMSE, assim, essa foi a resolução em bits utilizadas para a amostragem de sinais, que consequentemente aumentou o NMSE para -21,53 dB. Em seguida foi feito a modelagem do DPD em software o qual apresentou um comportamento inverso em relação ao do PA. Após isso foi feita a implementação do circuito em VHDL e validação em FPGA Virtex5 XC5VLX50T, que utilizou um total de 150 registradores, 692 LUTs e 4 unidades DSP48E, operando a uma frequência de 62,5 MHz. Em seguida seguiu-se para a etapa de síntese lógica a qual resultou em um circuito com 1567 células lógica, com uma area total de 28116 $ \mu m^2$ e um consumo de energia de 1.6 mW, atuando a uma taxa de operação de 33,34 MHz.Conclui-se, portanto, que o projeto alcançou os resultados esperados, com uma implementação eficaz do DPD em hardware, exceto pela taxa de operação da síntese lógica esperar-se uma melhor performance, o que não foi observado nas simulações.
