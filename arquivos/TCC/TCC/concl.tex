A evolução dos sistemas de comunicação sem fio tem promovido a implementação de diversos serviços móveis, tornando essencial que esses sistemas operem com máxima eficiência. Nesse cenário, a implementação de um DPD em cascata com o PA surge como uma alternativa de baixo custo e interessante para melhorar o desempenho desses sistemas.
O objetivo deste trabalho de conclusão de curso é implementar em hardware um DPD baseado no modelo de Polinômio de Memória. Para isso, o projeto foi dividido em quatro etapas: estudo do DPD e da modelagem matemática, modelagem do DPD em software, implementação do DPD em hardware e, finalmente, design do circuito integrado.
Sendo assim a primeira etapa de desenvolvimento do projeto foi a modelagem do PA em virgula flutuante, utilizando o método do MP, para fazer essa modelagem utilizou-se um polinômio de 2° grau com uma amostra de memória, para fazer a validação dessa modelagem uitlizou-se a métrica do NMSE. Nesta etapa obteve-se um NMSE de -23,57 dB, a próxima etapa consiste em otimizar a quantidade de células lógicas utilizadas no processo limitando o número de bits utilizados. Nesta etapa observou-se que a partir de 8 bits, não havia melhora expressiva no NMSE, assim, essa foi a resolução em bits utilizadas para a amostragem de sinais. E por fim foi feito a modelagem do DPD em software o qual apresentou um comportamento inverso em relação o do PA, assim satisfazendo as necessidades.  Atualmente, o projeto está na etapa 3, que corresponde à implementação em hardware.  
Conclui-se, portanto, que o projeto alcançou os resultados esperados, com uma modelagem eficaz do PA e do DPD, satisfazendo os critérios de desempenho estabelecidos nas etapas iniciais.