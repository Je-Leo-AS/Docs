A evolução dos sistemas de comunicação sem fio acarretou na implementação de diversas aplicações móveis e sem fio como desenvolvimento web, aplicação IoT, entre outros. Neste cenário, melhorar a eficiência energética se torna uma alternativa desejável tanto para os dispositivos móveis que buscam melhorar a autonomia das suas baterias, quanto para as estações de rádio base, que buscam reduzir sues desperdício em perdas de calor. No entanto, uma melhor eficiência energética implica em uma menor linearidade nos sistemas de amplificação de sinais, presentes nos sistemas transmissores de sinais de rádio. Isto é importante de ser ressaltado, pois a banda reservada para aplicações móveis é reduzida, de forma que para se alcançar maiores taxas de transmissão é necessário alternar estratégias de modulação tanto da fase, quanto da amplitude da onda portadora. E essas duas condições são conflitosas, já que a modulação AM é sensível a linearidade de forma que quanto mais linear um sistema ocorrem menos erros de transmissão. Sendo assim, uma alternativa para contornar esse obstáculo, que é implementar um sistema, eficiente energeticamente e linear é a implementação de um DPD em cascata com um PA. Portanto, o objetivo deste trabalho de conclusão de curso é o design de um circuito integrado dedicado de um DPD. Para atingir esse objetivo esse projeto foi divido em quatro etapas, o estudo e modelagem dos DPDs, modelagem do DPD em software, implementação do DPD em FPGA e finalmente o design do circuito integrado do DPD. Para a modelagem do DPD foi utilizada a métrica do NMSE, nela quanto menor o NMSE encontrado mais fiel e o modelo com a realidade. Sendo assim, na etapa de modelagem do PA alcançou-se um NMSE de -23.57 db. Em seguida, foi feita o levantamento do número de bits necessários para a realização desses cálculos de forma a minimizar o NMSE, para isso foi verificado que com apenas 8 bits de resolução do sinal já foi possível alcançar um NMSE próximo do valor alcançado em vírgula flutuante. Após feito esses levantamentos foi feita a implementação do circuito em VHDL e valida-lo em FPGA Virtex5 XC5VLX50T, que utilizou um total de 150 registradores, 692 LUTs e 4 unidades DSP48E, operando a uma frequência de 61,5 MHz. Em seguida seguiu-se para a etapa de síntese lógica a qual resultou em um circuito com 1567 células lógica, com uma area total de 28116 $um^2$ e um consumo de energia de 1.6 mW, atuando a uma faixa de operação de 50 MHz.

\textbf{Palavras-chave}: VHDL, FPGA, DPD 
