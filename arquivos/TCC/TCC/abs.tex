The evolution of wireless communication systems has led to the implementation of various mobile and wireless applications, such as web development and IoT applications, among others. In this context, improving energy efficiency becomes a desirable alternative both for mobile devices aiming to enhance battery autonomy and for base radio stations seeking to reduce heat loss waste. However, better energy efficiency implies lower linearity in the signal amplification systems present in radio signal transmitters. This is significant because the bandwidth allocated for mobile applications is limited, meaning that achieving higher transmission rates requires alternating modulation strategies for both the phase and amplitude of the carrier wave. These two conditions are conflicting since AM modulation is sensitive to linearity, and the more linear a system is, the fewer transmission errors occur. Thus, an alternative to overcoming this challenge—implementing a system that is both energy-efficient and linear—is the use of a Digital Predistortion (DPD) system in cascade with a Power Amplifier (PA). Therefore, the goal of this undergraduate thesis is the design of a dedicated integrated circuit for a DPD. To achieve this goal, the project was divided into four stages: studying and modeling DPDs, DPD modeling in software, DPD implementation on FPGA, and finally, designing the DPD integrated circuit. For DPD modeling, the NMSE (Normalized Mean Square Error) metric was used, where a lower NMSE indicates a model that is more faithful to reality. In the PA modeling stage, an NMSE of -23,57 dB was achieved. Next, the number of bits required to perform these calculations while minimizing NMSE was determined. It was found that with only 8 bits of signal resolution, it was already possible to achieve an NMSE close to the floating-point value. After this analysis, the circuit was implemented in VHDL and validated on an FPGA Virtex5 XC5VLX50T, using a total of 150 registers, 692 LUTs, and 4 DSP48E units, operating at a frequency of 61,5 MHz. Subsequently, the logical synthesis stage was carried out, resulting in a circuit with 1,567 logic cells, a total area of 28,116 $um^2$, and power consumption of 1,6 mW, operating at a frequency of 33,34 MHz.

\textbf{Palavras-chave}: VHDL, FPGA, DPD 