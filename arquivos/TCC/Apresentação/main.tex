\documentclass{if-beamer}
% --------------------------------------------------- %
%                  Presentation info	              %
% --------------------------------------------------- %
\title[Pré-distorcedor digital baseado em polinômio de memória]{\textbf{Projeto de um circuito integrado de um pré-distorcedor digital baseado em polinômio de memória}}
\subtitle{}
\author[Leonardo de Andrade Santos]{\large \negrito{Leonardo de Andrade Santos}}
\institute[UFPR/PR]{
    \small \textit{Universidade Federal do Paraná} \\
    %\textit{Uned Itaguaí}
}
\date{\the\year}
\logo{
\includegraphics[scale=.1, clip]{ufpr.png}
}
\subject{Presentation subject} % metadata
%trim={<left> <lower> <right> <upper>}
\graphicspath{{figuras/}}
% --------------------------------------------------- %
%                    Title + Schedule                 %
% --------------------------------------------------- %

\begin{document}

\begin{frame}
  \titlepage
\end{frame}

\begin{frame}{Programação}
  \tableofcontents
\end{frame}

\begin{frame}{Introdução}
\begin{itemize}
	\item Devido a evolução dos sistemas de comunicação moveis, que apresentam diversos serviços como aplicações multimídias, desenvolvimento web e aplicações IoT, aumentou a necessidade de desenvolver sistemas cada vez mais energeticamente eficientes;
	\item Largura de banda reservada para sistemas de comunicação sem fio reduzidas;
	\item Modulação na amplitude (AM) sensível a linearidade;
	\item Alternativa para contornar esses obstáculos é a implementação de um DPD em cascata ao Amplificador de Potência (PA);
\end{itemize}
\end{frame}

\begin{frame}{Objetivos}
	Objetivo deste trabalho é desenvolver um circuito integrado dedicado de um DPD.
	\begin{itemize}
		\item Realizar a modelagem de um PA e do DPD com o modelo de polinômio de memória;
		\item Implementar a descrição de hardware em 
		linguagem VHDL;
		\item Fazer o desing do circuito integrado utilizando as ferramentas de design do Cadence;
	\end{itemize}
\end{frame}

% --------------------------------------------------- %
%                      Presentation                   %
% --------------------------------------------------- %

%%%%%%%%%%%%%%%%%%%%%%%%%%%%%%%%%%%%%%%%%%%%%%%%%%%%%%%%%%%%%

\section{Fundamentação teórica}

%%%%%%%%%%%%%%%%%%%%%%%%%%%%%%%%%%%%%%%%%%%%%%%%%%%%%%%%%%%%%

\subsection{Sistema de comunicação}
\begin{frame}{Sistema de comunicação}
	\begin{itemize}
		\item A comunicação sem fio é dividida em 3 sub-sistemas principais:	o transmissor, o meio transmissor (ar), receptor;
		\item O PA é o componente de maior demanda energética por ser o componente que transfere potência da fonte para potência irradiada pela antena
	\end{itemize}
	\includegraphics[scale=0.5]{sistematrasmissorpng.png}
\end{frame}

%%%%%%%%%%%%%%%%%%%%%%%%%%%%%%%%%%%%%%%%%%%%%%%%%%%%%%%%%%%%%
\subsection{Problema}
\begin{frame}{Problema}

\begin{minipage}{.49\textwidth}

\begin{itemize}
    \item A curva amarela representa a restrição imposta pela norma regulamentadora;
    \item A curva em verde representa a resposta em frequência do sinal da amplificação;
    \item A curva em azul representa o sinal após a amplificação;
    \item As diferenças de densidade de potência nas bandas adjacentes ao canal representam a distorção causada pela não linearidade do PA;
    \item Maior eficiência implica em maior distorção do sinal.
\end{itemize}


\end{minipage}
\begin{minipage}{.49\textwidth}
	\includegraphics[scale=0.25]{normareg.png}
\end{minipage}

\end{frame}

%%%%%%%%%%%%%%%%%%%%%%%%%%%%%%%%%%%%%%%%%%%%%%%%%%%%%%%%%%%%%
\subsection{Mudar}
\begin{frame}{Mudar}
	
	\begin{minipage}{.49\textwidth}
		
		\begin{itemize}
			\item A característica de transferência não linear do PA 
			é caracterizada pela potência de saída que decai 1 dB da potência ideal, ou ponto de 1 dB de compressão de ganho (OCP1dB)
			\item  Efeito chamado memória causado devido aos componentes armazenadores (capacitâncias de energia e indutâncias), contribuindo significativamente na distorção.
			\item  O DPD, operando em banda base é uma solução eficiente com baixo custo computacional
		
		\end{itemize}
		
		
	\end{minipage}
	\begin{minipage}{.49\textwidth}
		\includegraphics[scale=0.5]{curvasaidaparf.png}
	\end{minipage}
	
\end{frame}


%%%%%%%%%%%%%%%%%%%%%%%%%%%%%%%%%%%%%%%%%%%%%%%%%%%%%%%%%%%%%
\subsection{Pré-distorcedor}
\begin{frame}{Pré-distorcedor}
\begin{minipage}{0.5\textwidth}
		
		\includegraphics[scale=0.5]{DPDCascata.png}
		
	\end{minipage}%
	\hspace{0.04\textwidth}
	\begin{minipage}{0.5\textwidth}
		\begin{itemize}
			\item De maneira sucinta, o DPD é conectado em 
			cascata ao PARF, e é projetado para apresentar a 
			função de transferência inversa ao PARF;
			\item Modelagem física: alto custo computacional;
			\item Modelagem matemática: baixo custo computacional;
			\item Se todos os parâmetros fossem conhecidos, conhecendo o equacionamento completo do circuito, uma função inversa poderia ser encontrada, possivelmente complexa como as séries de Volterra;
		\end{itemize}
	\end{minipage}
\end{frame}
%%%%%%%%%%%%%%%%%%%%%%%%%%%%%%%%%%%%%%%%%%%%%%%%%%%%%%%%%%%%%
\subsection{Séries de Volterra}
\begin{frame}{Séries de Volterra}
		\begin{itemize}
			\item As séries de Volterra são bastante difundidas para a modelagem comportamental;
			\item Não dependerem de parâmetros físicos do circuito; 
			\item Podem ser aplicados na modelagem de qualquer PA;
			\item Apenas medidas das informações de entrada (in) e saída (out) em domínio temporal são necessárias;
		\end{itemize}
		\vspace{0.05\textwidth}
			\[
		y(t) = h_0 + \int h_1(\tau_1) x(t - \tau_1) \, d\tau_1 + \int \int h_2(\tau_1, \tau_2) x(t - \tau_1) x(t - \tau_2) \, d\tau_1 \, d\tau_2 + \cdots
		\]
\end{frame}
%%%%%%%%%%%%%%%%%%%%%%%%%%%%%%%%%%%%%%%%%%%%%%%%%%%%%%%%%%%%%
\subsection{Polinômio de Memória}
\begin{frame}{Polinômio de Memória}
	\begin{minipage}{0.5\textwidth}
		\raggedleft
		\scriptsize
		\begin{equation}
			y(n) = \sum_{p=1}^{P} \sum_{m=0}^{M} h_{p,m} x(n-m) |x(n-m)|^{p-1}
		\end{equation}
			$
				\begin{bmatrix}
					y(1) \\
					y(2) \\
					y(3) \\
					y(4) \\
					y(5)
				\end{bmatrix}
				=
				\begin{bmatrix}
					x(1) & x(0) & x(1)x(1) & x(0)x(0) \\
					x(2) & x(1) & x(2)x(2) & x(1)x(1) \\
					x(3) & x(2) & x(3)x(3) & x(2)x(2) \\
					x(4) & x(3) & x(4)x(4) & x(3)x(3) \\
					x(5) & x(4) & x(5)x(5) & x(4)x(4)
				\end{bmatrix}
				\begin{bmatrix}
					h_{1,0} \\
					h_{1,1} \\
					h_{2,0} \\
					h_{2,1}
				\end{bmatrix}
			$
	
			
		
	\end{minipage}%
	\hspace{0.1\textwidth}
	\begin{minipage}{0.5\textwidth}
		\begin{itemize}
			\item Utilizado na modelagem comportamental simplificada das séries de Volterra.
			\item Considera apenas componentes unidimensionais2;
			\item Modelo compacto;
			\item Baixo custo computacional;
			\item Modelo linear nos coeficientes;
		\end{itemize}
	\end{minipage}
\end{frame}
%%%%%%%%%%%%%%%%%%%%%%%%%%%%%%%%%%%%%%%%%%%%%%%%%%%%%%%%%%%%%
\subsection{Field Port Gate Array (FPGA)}
\begin{frame}{Field Port Gate Array (FPGA)}
	\begin{minipage}{0.5\textwidth}
	
			\includegraphics[scale=0.5]{DPDCascata.png}
		
	\end{minipage}%
	\hspace{0.04\textwidth}
	\begin{minipage}{0.5\textwidth}
		\begin{itemize}
			\item FPGAs compõem uma classe de dispositivos lógicos programáveis;
			\item Eles possuem a capacidade de sintetizar arquiteturas complexas de eletrônica digital;
			\item São descritas como um conjunto de blocos digitais interconectados;
			\item Permite que tarefas possam ocorrer de forma paralela e sequencial;
			
		\end{itemize}
	\end{minipage}
\end{frame}
%%%%%%%%%%%%%%%%%%%%%%%%%%%%%%%%%%%%%%%%%%%%%%%%%%%%%%%%%%%%%
\subsection{Metodologia}
\begin{frame}{Metodologia}
	\begin{minipage}{0.5\textwidth}
		
		%\includegraphics[scale=0.5]{DPDCascata.png}
		
	\end{minipage}%
	\hspace{0.04\textwidth}
	\begin{minipage}{0.5\textwidth}
		O trabalho foi dividido em quatro etapas:
		\begin{itemize}
			\item Etapa 1: Estudos do DPD e modelagem matemática do PA.
			\item Etapa 2: Implementação do DPD em software.
			\item Etapa 3: Implementação do DPD em FPGA.
			\item Etapa 4: Simulação em FPGA.
		\end{itemize}
	\end{minipage}
	\begin{minipage}{0.5\textwidth}
		
		\includegraphics[scale=0.1]{diagrama.png}
		
	\end{minipage}%
\end{frame}
%%%%%%%%%%%%%%%%%%%%%%%%%%%%%%%%%%%%%%%%%%%%%%%%%%%%%%%%%%%%%
\subsection{Dados Utilizados}
\begin{frame}{Dados Utilizados}
	\begin{minipage}{0.5\textwidth}
		
		
	\end{minipage}%
	\hspace{0.04\textwidth}
	\begin{minipage}{0.5\textwidth}
		\begin{itemize}
			\item Amplificador de potência classe AB, HEMT (transistor de efeito de campo de heterojunção) fabricado em tecnologia GaN.
			\item Excitado por um sinal portadora de frequência de 900	MHz.
			\item Modulado por um sinal de envoltória WCDMA 3GPP 3,84 MHz de largura de banda.
			\item Os dados de entrada e saída do amplificador de potência foram medidos usando um analisador de sinal vetorial (VSA) Rohde \& Schwarz FSQ com uma taxa de amostragem de 61,44 MHz.
		\end{itemize}
	\end{minipage}
	\begin{minipage}{0.5\textwidth}
		
		\includegraphics[scale=0.4]{analisador.jpeg}
		
	\end{minipage}%
\end{frame}

\section{Implementação e software}
\subsection{Modelagem do PA}
%%%%%%%%%%%%%%%%%%%%%%%%%%%%%%%%%%%%%%%%%%%%%%%%%%%%%%%%%%%%%
\begin{frame}{Modelagem do PA}
	
	
	\begin{minipage}{0.5\textwidth}
		\begin{itemize}
			\item  Implementação em Python;
			\item Modelagem do PA, com cálculo em vírgula flutuante;
			\item NMSE de -23,57 dB, para um Polinômio de $2^\circ$ grau com uma amostra de memória;
		\end{itemize}
	\end{minipage}%
	\hspace{0.04\textwidth}
	\begin{minipage}{0.5\textwidth}
		
		\includegraphics[scale=0.4]{modeloPA.png}
		
	\end{minipage}%
\end{frame}
%%%%%%%%%%%%%%%%%%%%%%%%%%%%%%%%%%%%%%%%%%%%%%%%%%%%%%%%%%%%%
\begin{frame}{Ajuste da Resolução do Sinal}
	\begin{minipage}{0.5\textwidth}
		\begin{itemize}
			\item Adaptação para realização dos cálculos em vírgula fixa, com uma resolução N de bits;
			\item Inicialmente realizado uma normalização dos dados e em seguida é feitos os cálculos em vírgula fixa Dados DPD com polinômio de memória de grau 2 com um sinal de memória
			\item  Dados DPD com polinômio 	de memória de grau 2 com 
			um sinal de memória
		\end{itemize}
	
		
	\end{minipage}%
	\hspace{0.04\textwidth}
	\begin{minipage}{0.5\textwidth}
	\includegraphics[scale=0.25]{bits.png}
	\end{minipage}
\end{frame}
%%%%%%%%%%%%%%%%%%%%%%%%%%%%%%%%%%%%%%%%%%%%%%%%%%%%%%%%%%%%%
\subsection{Modelagem do DPD}
\begin{frame}{Modelagem do DPD}
	\begin{minipage}{0.5\textwidth}
		\begin{itemize}
			\item Cálculo do modulo DPD da mesma forma que do PA, apenas invertendo os dados de entrada com os de saída.
		\end{itemize}
		
		
	\end{minipage}%
	\hspace{0.04\textwidth}
	\begin{minipage}{0.5\textwidth}
		\includegraphics[scale=0.40]{modelodpd.png}
	\end{minipage}
\end{frame}
\section{Implementação em Hardware}
%%%%%%%%%%%%%%%%%%%%%%%%%%%%%%%%%%%%%%%%%%%%%%%%%%%%%%%%%%%%%
\subsection{Desenvolvimento FPGA}
\begin{frame}{Desenvolvimento do VHDL}
	
		\includegraphics[scale=0.2]{diagrama_process.png}
\end{frame}
%%%%%%%%%%%%%%%%%%%%%%%%%%%%%%%%%%%%%%%%%%%%%%%%%%%%%%%%%%%%%
\begin{frame}{Fluxo de cálculo}
	\includegraphics[scale=0.25]{fluxo_de_calculo.png}
\end{frame}
%%%%%%%%%%%%%%%%%%%%%%%%%%%%%%%%%%%%%%%%%%%%%%%%%%%%%%%%%%%%%
\begin{frame}{Resultado da simulação da FPGA}
	\begin{minipage}{0.5\textwidth}
		\begin{itemize}
			\item FPGA Virtex5 XC5VLX50T;
			\item frequência de operação 62,5 MHz.
		\end{itemize}
		\begin{tabular}{|l|r|r|}
			\hline
			Recursos & Unidade & Percentual \\
			\hline
			Registradores & 150 & 1\% \\
			LUTs & 692 & 2\% \\
			DSP48E & 4 & 8\% \\
			\hline
		\end{tabular}
		
		
	\end{minipage}%
	\hspace{0.04\textwidth}
	\begin{minipage}{0.5\textwidth}
		\includegraphics[scale=0.40]{fpgasim.png}
	\end{minipage}
\end{frame}
%%%%%%%%%%%%%%%%%%%%%%%%%%%%%%%%%%%%%%%%%%%%%%%%%%%%%%%%%%%%%
\subsection{Desenvolvimento Tecnologia BiCMOS 8HP 130 nm}
\begin{frame}{Síntese Lógica}
		\begin{minipage}{0.5\textwidth}
		\begin{itemize}
			\item Tecnologia 8HP 130nm.
		\end{itemize}
		\begin{tabular}{|l|r|r|}
			\hline
			Recursos & Quantidade  \\
			\hline
			Células lógica & 1567  \\
			Consumo & 1.6 mW  \\
			Área & 28116 $um^2$  \\
			\hline
		\end{tabular}
		
	\end{minipage}%
	\hspace{0.04\textwidth}
	\begin{minipage}{0.5\textwidth}
		\includegraphics[scale=0.10]{Sintesi-lógica.png}
	\end{minipage}
\end{frame}
%%%%%%%%%%%%%%%%%%%%%%%%%%%%%%%%%%%%%%%%%%%%%%%%%%%%%%%%%%%%%
\begin{frame}{Resultado simulação da Síntese Lógica}
	\begin{minipage}{0.5\textwidth}
		\begin{itemize}
			\item Frequência de operação 33,34 MHz.
			\item Mesmo Testbench utilizado na FPGA.
			%\includegraphics[width=0.1]{NcLaunch.png}
		\end{itemize}
		
	\end{minipage}%
	\hspace{0.04\textwidth}
	\begin{minipage}{0.5\textwidth}
		\includegraphics[scale=0.40]{simpossin.png}
	\end{minipage}
\end{frame}

%%%%%%%%%%%%%%%%%%%%%%%%%%%%%%%%%%%%%%%%%%%%%%%%%%%%%%%%%%%%%
\begin{frame}{Conclusão}
	A evolução dos sistemas de comunicação sem fio exige eficiência máxima, e a implementação de um Pré-Distorcedor Digital (DPD) em cascata com o Amplificador de Potência (PA) oferece uma solução de baixo custo para melhorar o desempenho. Este trabalho validou um código em linguagem de descrição de hardware capaz de processar em tempo real as características não lineares e efeitos de memória de um amplificador, minimizando o uso de recursos lógicos e o consumo de energia. Esse trabalho foi validado em uma  FPGA Virtex5 XC5VLX50T, utilizando um total de 150 registradores, 692 LUTs e 4 DSP48Es, operando a uma frequência de 62,5 MHz.  
\end{frame}
%%%%%%%%%%%%%%%%%%%%%%%%%%%%%%%%%%%%%%%%%%%%%%%%%%%%%%%%%%%%%
\section*{Referências}
\begin{frame}{Referências}
	\setbeamertemplate{navigation symbols}{}
	
		\begin{thebibliography}{}
		\bibitem{John2016} Elton John, ``Modelagem comportamental de amplificadores de potência de radiofrequência usando termos unidimensionais e bidimensionais de séries de Volterra", 2016.
		\bibitem{Kenington2000} Peter Kenington, ``High Linearity RF Amplifier Design", 2000.
		\bibitem{Cripps2006} Steve Cripps, ``RF Power Amplifiers for Wireless Communications", 2006.
		\bibitem{Chavez2018} Joel Huanca Chavez, ``Estudo comparativo entre as arquiteturas de identificação de pré-distorcedores digitais através das aprendizagens direta e indireta", 2018.
		\bibitem{Pedroni2010} Volnei Pedroni, ``Eletrônica Digital e VHDL ", 2010.
		\bibitem{Gonçalves2009} Eduardo Gonçalves de Lima and Giovanni Ghione, ``Behavioral modeling and digital base-band predistortion of RF power amplifiers", 2009.
		\bibitem{Schuartz2017} Luis Schuartz and Eduardo Lima, ``Polinômios com Memória de Complexidade Reduzida e sua Aplicação na Pré-distorção Digital de Amplificadores de Potência", 2017.
		
		\bibitem{Bonfim2016} Elton J Bonfim and Eduardo G De Lima, ``A Modified Two Dimensional Volterra-Based Series for the Low-Pass Equivalent Behavioral Modeling of RF Power Amplifiers", vol. 47, pp. 27-35, 2016.
		
		%	\bibitem{Wayne} Wayne Wolf, ``Modern VLSI Design: IP-Based Design, Fourth Edition", Prentice Hall Modern Semiconductor Design Series.
		%	\bibitem{Raychaudhuri2012} Dipankar Raychaudhuri and Narayan B. Mandayam, ``Frontiers of Wireless and Mobile Communications", Proceedings of the IEEE, vol. 100, no. 4, pp. 824-840, 2012.
		%	\bibitem{Silva2013} Pedro Silva, ``Combinação entre pré-distorção digital e redução de fator de crista para linearização de amplificadores de potência para sistemas de telecomunicações móveis", 2013.
		%	\bibitem{Lima2009} Eduardo Gonçalves de Lima and Giovanni Ghione, ``Behavioral modeling and digital base-band predistortion of RF power amplifiers", 2009.
		%	\bibitem{CMOS2010} Neil H.E.Weste and David Money Harris, ``CMOS VLSI Design: A Circuits and Systems Perspective (4th Edition)", 2010.
		
		%	\bibitem{Pedroni2020} Volnei Pedroni, ``Circuit design with VHDL", 2020.
	\end{thebibliography}

\end{frame}
%%%%%%%%%%%%%%%%%%%%%%%%%%%%%%%%%%%%%%%%%%%%%%%%%%%%%%%%%%%%%


%%%%%%%%%%%%%%%%%%%%%%%%%%%%%%%%%%%%%%%%%%%%%%%%%%%%%%%%%%%%%

\end{document}
