\documentclass[a4paper,12pt]{article}
\usepackage[utf8]{inputenc}
\usepackage[T1]{fontenc}
\usepackage{geometry}
\geometry{left=2.5cm,right=2.5cm,top=2.5cm,bottom=2.5cm}
\usepackage{setspace}
\usepackage{titlesec}
\usepackage{parskip}
\usepackage{times}
\usepackage[portuguese]{babel}
\usepackage{csquotes}
\usepackage{bibentry}
\usepackage[alf, abnt-repeated-title-omit=yes]{abntex2cite}
% Configuring section titles
\titleformat{\section}{\large\bfseries\centering}{\thesection}{1em}{}
\titlespacing*{\section}{0pt}{12pt}{6pt}

% Remove page numbers
\pagestyle{empty}

\begin{document}

% Cover Page
\begin{titlepage}
    \centering
    {\normalsize \textbf{PPGEE-UFPR}\par}
    \vspace{0.1cm}
    {\normalsize EELT7019 - Inteligência Artificial Aplicada\par}
    \vspace{9cm}
    {\normalsize  \textbf{Lista 05 - Levantamento Bibliográfico}\par}
    \vspace{9cm}
    {\normalsize  \textbf{Professor:} Dr. Alexandre Rasi Aoki\par}
    \vspace{1cm}
    {\normalsize  Curitiba, 2020\par}
\end{titlepage}

% Introduction Page
\clearpage
\section{Introdução}
\begin{spacing}{1.5}

Esta introdução detalha o processo de seleção de um portfólio bibliográfico, conforme as diretrizes para o levantamento bibliográfico solicitado. O trabalho iniciou-se com a consulta ao Portal de Periódicos da CAPES e, em seguida, ao Webqualis, onde foi baixado o arquivo que listava as revistas classificadas na área de Engenharias IV.

A partir desta lista, foi realizada uma pesquisa aprofundada por artigos científicos publicados em periódicos internacionais relevantes. A pesquisa focou em publicações no período entre 2017 e 2020, com temas especificamente relacionados à Inteligência Artificial (IA) e Agricultura. Esta escolha temática é de grande relevância e alinhamento com a minha área de atuação profissional, uma vez que a empresa em que trabalho atua na área de agricultura de precisão, que busca otimizar a produtividade e a sustentabilidade no setor agrícola através de tecnologias avançadas.

Dentre os periódicos pesquisados e artigos relevantes identificados, a revista Sensors, publicada pelo Multidisciplinary Digital Publishing Institute, destacou-se. O artigo selecionado foi o "Sensors Driven AI-Based Agriculture Recommendation Model for Assessing Land Suitability", publicado em 2019. Verificou-se que a revista Sensors (Basel), com ISSN 1424-8220, possui classificação 2 na área de Engenharias IV no Webqualis.
\end{spacing}

% Summary Page
\clearpage
\section{Resumo}
\begin{spacing}{1.5}

O artigo "Sensors Driven AI-Based Agriculture Recommendation Model for Assessing Land Suitability", publicado na revista Sensors em 2019, apresenta um modelo de recomendação baseado em inteligência artificial (IA) e sensores para avaliar a adequação de terras agrícolas. Com a projeção de crescimento populacional de dois bilhões até 2050 e o aumento limitado de áreas cultiváveis, a pesquisa destaca a necessidade de técnicas agrícolas inteligentes para aumentar a produtividade. O estudo propõe um sistema especialista que integra redes de sensores com redes neurais artificiais, especificamente o modelo Multi-Layer Perceptron (MLP), para classificar terras agrícolas em quatro categorias: mais adequada, adequada, moderadamente adequada e inadequada.

A metodologia utiliza sensores IoT, como sensores de pH, umidade do solo, salinidade e eletromagnéticos, para coletar dados de parâmetros como textura do solo, drenagem interna, capacidade de troca catiônica, matéria orgânica e salinidade. Esses dados, coletados de terras em Vellore e Tiruvannamalai, na Índia, foram armazenados na nuvem AWS via Raspberry Pi e processados para treinar o modelo de IA. O conjunto de dados, composto por 1000 instâncias com 14 atributos, foi dividido em 75\% para treinamento e 25\% para teste, com normalização para lidar com valores heterogêneos. O modelo MLP com quatro camadas ocultas demonstrou maior eficácia na classificação multiclasse em comparação com redes neurais tradicionais e MLP com três camadas, alcançando até 99,9\% de acurácia com 80 neurônios por camada.

Os resultados mostram que o modelo MLP com quatro camadas ocultas supera outros modelos, especialmente com maior número de neurônios, reduzindo erros (MSE e RMSE) e melhorando métricas como precisão, recall e F1-score. A classificação multiclasse permite aos agricultores identificar terras prontas para cultivo, que necessitam de melhorias ou inadequadas, minimizando perdas e otimizando a produção. O sistema é escalável, podendo ser aplicado a diferentes regiões, e representa uma inovação em relação a modelos binários tradicionais, oferecendo recomendações mais precisas.

A pesquisa enfatiza a integração de IoT e IA como uma solução econômica e eficiente para a agricultura de precisão, especialmente em regiões de baixa renda. O modelo proposto contribui para o desenvolvimento agrícola sustentável, fornecendo uma ferramenta prática para tomada de decisão baseada em dados em tempo real. Conforme destacado por \cite{Vincent2019}, a abordagem combina sensores e IA para oferecer uma classificação detalhada da adequação do solo, promovendo maior produtividade e sustentabilidade na agricultura.



\end{spacing}

% References Page
\clearpage


\bibliography{artigo.bib}

\end{document}