\subsection{Série de Volterra}
A série de Volterra é uma generalização multidimensional da série de Taylor, aplicada a sistemas dinâmicos. A modelagem começa com a representação do sistema por uma série infinita de integrais convolucionais, onde cada termo representa uma ordem de não linearidade e memória. A saída \( y(t) \) do sistema pode ser expressa pela equação \ref{eq:Volterra}:

\[
y(t) = h_0 + \sum_{n=1}^{\infty} \int_{-\infty}^{\infty} \cdots \int_{-\infty}^{\infty} h_n(\tau_1, \tau_2, \ldots, \tau_n) \prod_{i=1}^{n} x(t - \tau_i) \, d\tau_i
\label{eq:Volterra}
\]

onde \( h_n \) são os núcleos de Volterra, caracterizando a resposta do sistema para a \( n \)-ésima ordem de não linearidade, e \( x(t) \) é a entrada do sistema. Em muitas aplicações práticas, a série é truncada para incluir um número finito de termos devido à impraticabilidade de identificar todos os núcleos de uma série infinita \cite{Gonçalves2009}.
\subsection{Polinômio de memória}
O Polinômio de Memória (MP) é um modelo simples, utilizado na modelagem simplificada das séries de Volterra, considerando componentes unidimensionais. Este modelo é compacto, de baixo custo computacional e linear em seus parâmetros, sendo eficaz para amplificadores de potência (PAs) com pouco efeito de memória. Como o DPD e o pós-distorcedor apresentam características inversas às do PA, o mesmo modelo pode ser utilizado. A equação \ref{eq:mp} descreve o MP conforme ilustrado por \cite{Schuartz2017}:

\[
y(n) = \sum_{p=1}^{P} \sum_{m=0}^{M} h_{p,m} x(n - m) \left| x(n - m) \right|^{p-1}
\label{eq:mp}
\]

Para implementar esse modelo em hardware, é necessário paralelizar as operações aritméticas, visando altas taxas de operação. Nesse sentido, as FPGAs são uma alternativa viável para a implementação de circuitos pré-distorcedores.