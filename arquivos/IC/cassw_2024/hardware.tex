Essa etapa consiste na implementação do DPD em FPGA. Para isso, é necessário realizar paralelizações nas operações aritméticas. Para isso o processo  A cada ciclo, duas operações são realizadas em paralelo: o sinal atual é elevado ao quadrado e registrado, enquanto ocorre a multiplicação da matriz de extração pelos seus respectivos coeficiente e em um terceiro processo é realizado o somatório desses sinais multiplicados. Esse processo ocorre P vezes para os \( P \) graus do polinômio de memória. Portanto, a saída do DPD é incompleta para os primeiros \( P \) períodos de clock, pois, nesses primeiros ciclos, realiza-se o cálculo com base em entradas de sinais anteriores que ainda não ocorreram, resultando em uma saída incompleta.


