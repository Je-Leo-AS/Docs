Essa etapa consiste na implementação do DPD em FPGA, onde é necessário realizar paralelizações nas operações aritméticas, separando os processos de potência, multiplicação e soma em etapas distintas. A cada ciclo de clock, duas operações são realizadas em paralelo: o sinal atual é elevado ao quadrado e registrado, enquanto ocorre o somatório do produto entre os sinais do mesmo instante de tempo e seus respectivos coeficientes. Esse processo é repetido P vezes, de acordo com os \( P \) graus do polinômio de memória. Como resultado, a saída do DPD é incompleta durante os primeiros \( P \) ciclos de clock, pois, nesse período inicial, o cálculo é realizado com base em entradas de sinais anteriores que ainda não ocorreram, resultando em uma saída incompleta.