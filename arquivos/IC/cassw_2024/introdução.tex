A evolução dos sistemas de comunicação móveis, impulsionada pela demanda por serviços como multimídia, desenvolvimento web e IoT, trouxe desafios, como a necessidade de melhorar a eficiência energética de dispositivos móveis e estações de rádio base \cite{John2016}. Para atender a essas demandas, estratégias de modulação que alteram fase e amplitude das ondas portadoras se tornaram essenciais \cite{Kenington2000}. A modulação em amplitude requer linearidade na transmissão para evitar erros e interferências \cite{Cripps2006}, o que representa um desafio para o projetista do PARF, que deve desenvolver um hardware eficiente e linear ao mesmo tempo, objetivos que podem entrar em conflito \cite{Chavez2018}. Uma solução é o pré-distorcedor de sinais digitais em banda base, que compensa a distorção do PARF \cite{Cripps2006}. O DPD precisa de um modelo preciso e de baixa complexidade computacional, e pode ser modelado de forma física ou empírica, sendo a segunda opção mais simples, mas com menor precisão. Para atender aos rigorosos requisitos de frequência de operação, a paralelização das operações torna-se essencial, e as FPGAs surgem como uma alternativa viável \cite{Pedroni2010}. Neste contexto, o objetivo deste projeto é criar e validar um código em linguagem de descrição de hardware capaz de processar, em tempo real, as características de transferência inversa de um amplificador, minimizando o uso de recursos lógicos e o consumo de energia. O projeto inclui modelar o amplificador e o pré-distorcedor digital em software e implementar o DPD em hardware.