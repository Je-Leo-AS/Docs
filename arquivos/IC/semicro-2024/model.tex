\subsection{Séries de Volterra}

Segundo \cite{Gonçalves2009} série de Volterra pode ser vista como uma extensão multidimensional da série de Taylor para sistemas dinâmicos. A modelagem começa com a representação do sistema através de uma série infinita de integrais convolucionais, onde cada termo da série corresponde a uma ordem de não linearidade e memória.

A saída \( y(t) \) de um sistema pode ser expressa pela equação \ref{eq:Volterra}: \begin{equation}
	y(t) = h_0 + \sum_{n=1}^{\infty} \int_{-\infty}^{\infty} \cdots \int_{-\infty}^{\infty} h_n(\tau_1, \tau_2, \ldots, \tau_n) \prod_{i=1}^{n} x(t - \tau_i) \, d\tau_i
	\label{eq:Volterra}
\end{equation} onde \( h_n \) são os núcleos de Volterra, que caracterizam a resposta do sistema para a \( n \)-ésima ordem de não linearidade e \( x(t) \) é a entrada do sistema.

Os núcleos de Volterra \( h_n \) são funções de várias variáveis que capturam a dinâmica do sistema em diferentes ordens. Para a maioria das aplicações práticas, a série é truncada para incluir apenas um número finito de termos, já que a identificação de todos os núcleos de uma série infinita é impraticável.

\subsection{Polinômio de memória}

Um modelo simples, utilizado na modelagem comportamental simplificada das séries de Volterra considerando apenas componentes unidimensionais\footnote{Cada termo do somatório é composto por amostras no mesmo instante, por exemplo: $x(n)|x(n)|,x(n - 1)|x(n - 1)|$; termos bidimensionais são compostos por amostras em instantes de tempos distintos, como por exemplo: $x(n)|x(n - 1)|$} é o MP, que é um modelo compacto, de baixo custo computacional e linear em seus parâmetros. O MP gera baixo erro quando aplicado à PAs que apresentam pouco efeito de memória. O DPD e pós distorsor apresentam característica inversa a do PA \cite{Schuartz2017}, portanto o mesmo modelo pode ser utilizado. A equação \ref{eq:mp} apresenta o MP conforme é ilustrado por \cite{Schuartz2017}: 

\begin{equation}
	y(n) = \sum_{p=1}^{P} \sum_{m=0}^{M} h_{p,m} x(n - m) \left| x(n - m) \right|^{p-1}
	\label{eq:mp}
\end{equation}

Como a proposta do trabalho é a implementação em hardware desse modelo, torna-se necessário paralelizar operações aritméticas de forma a alcançar uma taxa de operação que satisfaça a norma regulamentadora. Nesse contexto, as FPGAs apresentam-se como uma alternativa viável para a implementação de circuitos pré-distorcedores.