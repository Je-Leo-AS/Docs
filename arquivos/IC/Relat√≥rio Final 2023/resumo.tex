A evolução da comunicação sem fio permitiu o desenvolvimento de diversas aplicações móveis, como IoT. A melhoria da eficiência energética é essencial para dispositivos móveis e estações de rádio, buscando reduzir perdas e aumentar a autonomia. Porém a eficiência energética reduz a linearidade dos amplificadores, o que afeta modulações de fase e amplitude, aumentando erros de transmissão. Sendo assim, uma alternativa para contornar esse obstáculo, que é implementar um sistema, eficiente energeticamente e linear é a implementação de um (DPD) em cascata com um Amplificador de Potência (PA). Portanto, este trabalho visa implementar um código em VHDL para processar as características de um amplificador em tempo real, minimizando não linearidades e consumo de energia. A modelagem foi feita inicialmente em software e implementada em FPGA (Field-Programmable Gate Array) para validação e análise de desempenho. Durante o desenvolvimento, testes e análises foram realizados com um polinômio de memória de segundo grau e um sinal de amostra memorizado, mostrando uma redução efetiva nas distorções do PA. Foi utilizada a métrica de Erro Quadrático Médio Normalizado (NMSE - Normalized Mean Squared Error) para quantificar a qualidade da modelagem implementada. O NMSE de -23,57 dB mostrou eficiência na redução de distorções. Valores menores de NMSE refletem melhor capacidade de reduzir distorções e preservação da fidelidade do sinal. Assim a próxima etapa consiste na modelagem do PA em vírgula fixa, a fim de atingir o NMSE mais próximo do atingido em vírgula flutuante com a menor resolução em bits, visando alcançar a melhor desempenho em hardware. A modelagem em vírgula fixa com 8 bits otimizou o desempenho. Com base nos resultados obtidos e no desempenho promissor alcançado, espera-se que, em um trabalho futuro, o projeto de um circuito integrado dedicado para um DPD, modelado com um polinômio de memória de segundo grau e uma amostra armazenada, possa fornecer uma solução eficiente e robusta para a pré-distorção digital em amplificadores de potência, promovendo a evolução dos sistemas de comunicação móvel.
\\
\\
\textbf{Palavras-chave}: VHDL, FPGA, DPD 
