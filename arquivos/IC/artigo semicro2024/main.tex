\documentclass[twocolumn, a4paper]{article}
\fontsize{10.5}{10}\selectfont
\def\figsize{0.40\textwidth} 
% ------------------------------------------------------------------
% Importação do arquivo de definições
\usepackage{graphicx} % pacote para adição de figuras no texto
\usepackage{multirow} % textos e tabelas com multiplas linhas
\usepackage{multicol} % textos e tabelas com multiplas colunas
\usepackage{cuted} % definir espaçamento do titulo
\usepackage{fancyhdr} % pacote usado para o rodapé
\usepackage{mathptmx} % Fonte da família Times
\usepackage{titling} % ajuste de título e subtitulos
\usepackage{titlesec} % ajuste de título e subtitulos

%----------------------------------------------------------------------
% formatação em itálico da legenda da figura
\usepackage{caption}
\captionsetup[figure]{format=plain, textfont=it}{}
% --------------------------------------------------------------------
% Define o espaçamento superior antes do início do título
\setlength{\droptitle}{-6em} 
%----------------------------------------------------------------------
%----------------------------------------------------------------------
% Definição do texto e numeração das seções e legenda de figuras e tabela
\titlelabel{\thetitle.\vspace{2pt}}
\titleformat{\section}
  {\normalfont\large\bfseries\filcenter}{\thesection.}{0.5em}{}
\titleformat{\subsection}
  {\normalfont\normalsize\itshape}{\thesubsection.}{0.5em}{}
\titlespacing{\section}{0pt}{10pt}{5pt}

\renewcommand{\thesubsection}{\Alph{subsection}}
\renewcommand{\thesection}{\Roman{section}}
\renewcommand{\figurename}{Fig.}
\renewcommand{\tablename}{TABELA}
%----------------------------------------------------------------------
% Retira a numeração automática de rodapé
\pagenumbering{gobble}
%----------------------------------------------------------------------
% Rodapé da primeira página
\fancypagestyle{specialfooter}{%
  
  \fancyhf{}
  \renewcommand\headrulewidth{0pt}
  \fancyfoot[C]{IEEE Circuits and Systems Workshop \\
Curitiba, Brasil}
}
%----------------------------------------------------------------------
%----------------------------------------------------------------------
% Definição da margem das páginas
\usepackage{geometry}
 \geometry{
     a4paper,
     left=16mm,
     top=19mm,
     right=16mm,
     bottom=25.4mm
 }
%----------------------------------------------------------------------
% ---------------------------------------------------------------------
% Titulo do texto
\graphicspath{{Figuras/}}
\title{
    \begin{minipage}[c]{0.3\textwidth} 
          \includegraphics[width=\textwidth]{Semicro_title.png} 
    \end{minipage}\begin{minipage}[c]{0.7\textwidth} 
             \begin{center}
                \Huge
                Título do artigo\\
                \vspace{20pt}
                \Large
                  Primeiro autor\textsuperscript{1}, Segundo autor\textsuperscript{1}, Terceiro autor\textsuperscript{2} \\
                  \normalsize
                  \textsuperscript{1}Afiliação 1, cidade, país \\
                  \textsuperscript{2}Afiliação 2, cidade, país \\
                  E-mail de contato
             \end{center}
    \end{minipage}
}

% Define o tamanho do espaço vertical abaixo do título
\date{\vspace{-8em}}
% Define a separação entre as colunas
\setlength\columnsep{2em}



\usepackage[portuguese]{babel}
%---------------------------
% ------------------------------------------------------------------
% Início do documento
\begin{document}
% ------------------------------------------------------------------
% Definição da parte do titulo do arquivo 
\begin{strip}
  \vspace*{\dimexpr-\baselineskip-\stripsep\relax}
  %\centering
   \maketitle
  \vskip\baselineskip
\noindent %\makebox[\textwidth]{\rule{1.1\paperwidth}{0.4pt}} Linha horizontal
  \vskip\baselineskip
\end{strip}
% EDITAR A FIGURA 1  AUMENTAR A FIGURA APROXIMANDO OS BLOCOS
%CORRIGR NUMERO DE PAGINAS
%TIRAR FONTE : AUTOR
%REORDENAR LISTAS DE REFERÊNCIAS CONFORME APARECE NO TEXTO
% ------------------------------------------------------------------
% Nota de rodarezzé da primeira página
\thispagestyle{specialfooter}
% ------------------------------------------------------------------
% Resumo (definição em itálico e negrito)
\vspace{3pt}\textbf{\textit{Resumo - A evolução da comunicação sem fio impulsionou aplicações como IoT, onde a eficiência energética é crucial para dispositivos móveis e estações de rádio. No entanto, essa eficiência compromete a linearidade dos amplificadores de potência (PA), afetando modulações de fase e amplitude. Sendo assim, uma alternativa para contornar esse obstáculo é a implementação de um Pré-distorcedor digital (DPD) em cascata com um PA. Portanto, este trabalho propõe a implementação de um código em VHDL para processar as características de transferências inversa de um amplificador de potência em tempo real, minimizando não linearidades e consumo de energia. A modelagem foi feita inicialmente em software e implementada em FPGA (Field-Programmable Gate Array) para validação e análise de desempenho. Durante o desenvolvimento, testes e análises foram realizados com um polinômio de memória de segundo grau e um sinal de amostra memorizado, mostrando uma redução efetiva nas distorções do PA. Foi utilizada a métrica de Erro Quadrático Médio Normalizado (NMSE - Normalized Mean Squared Error) para quantificar a qualidade da modelagem implementada. O NMSE de -23,57 dB mostrou eficiência na modelagem das distorções inversas. Assim a etapa seguinte consistiu na modelagem do PA em vírgula fixa, a fim de atingir o NMSE mais próximo do atingido em vírgula flutuante com a menor resolução em bits, visando alcançar o melhor desempenho em hardware. A modelagem em vírgula fixa com 8 bits aprimorou o desempenho. Por fim, o modelo foi implementado em hardware na FPGA Virtex5 XC5VLX50T, utilizando um total de 150 registradores, 692 LUTs e 4 unidades DSP48E, operando a uma frequência de 61,5 MHz.}}
% ------------------------------------------------------------------

\section{INTRODUÇÃO}
A evolução dos sistemas de comunicação móveis, impulsionada pela crescente demanda por comunicações mais rápidas e eficientes, tem levado à implementação de uma variedade de serviços, incluindo aplicações multimídia, desenvolvimento web e aplicações IoT \cite{John2016}. No entanto, essa evolução também trouxe desafios significativos, como a necessidade de melhorar a eficiência energética, tanto para dispositivos móveis, visando aumentar a autonomia da bateria, quanto para estações de rádio base, visando reduzir o consumo de energia devido às perdas de calor. Para atender a essas demandas, estratégias de modulação que alteram tanto a fase quanto a amplitude de ondas portadoras em radiofrequência se tornaram essenciais \cite{Kenington2000}. Além disso, a modulação na amplitude requer linearidade na transmissão para evitar erros e interferências na comunicação entre usuários vizinhos \cite{Cripps2006}. Essa complexa tarefa recai sobre o projetista do PARF (Amplificador de Potência de Rádio Frequência), que enfrenta o desafio de desenvolver um hardware eficiente em termos energéticos e linear ao mesmo tempo, uma vez que esses dois objetivos podem entrar em conflito \cite{Chavez2018}. Uma solução para contornar esse desafio é a implementação de um pré-distorcedor de Sinais Digital em Banda Base, que visa compensar a distorção causada pelo PARF \cite{Cripps2006}. O DPD (Pre-distorcedor Digital) é conectado em cascata ao PARF e requer um modelo de alta precisão e baixa complexidade computacional para representar as características de transferência direta e inversa do PARF. Existem duas abordagens para modelar o PARF: modelos físicos, que são detalhadas e computacionalmente complexos, e modelos empíricos, que se baseiam em medições de entrada e saída do PARF, com menor complexidade computacional, mas com uma possível diminuição da precisão. Devido às exigências rigorosas de frequência de operação, a paralelização das operações torna-se essencial, e as FPGAs (Matriz de Portas Programáveis em Campo) emergem como uma alternativa viável para a implementação de circuitos pré-distorcedores \cite{Pedroni2010}. As FPGAs são dispositivos lógicos programáveis que permitem a reconfiguração física de componentes de eletrônica digital, acelerando processos e suportando operações paralelas e sequenciais. 
Neste contexto, este projeto foi planejado com o objetivo geral de construir e validar um código em linguagem de descrição de hardware capaz de processar, em tempo real, as características de transferência de um amplificador, reproduzindo suas não linearidades e efeitos de memória. Além disso, o código deve ocupar a quantidade mínima de recursos lógicos do circuito digital que irá sintetizá-lo, visando à redução do consumo de energia. Para atingir esse objetivo, o trabalho foi desenvolvido com base nos seguintes objetivos específicos: modelar com precisão o amplificador de potência (PA) em software; modelar o pré-distorcedor digital (DPD) em software a partir da modelagem do PA; e implementar o DPD em hardware utilizando uma linguagem de descrição de hardware (HDL).



\section{MODELAGEM MATEMÁTICA}
\subsection{Série de Volterra}
A série de Volterra é uma generalização multidimensional da série de Taylor, aplicada a sistemas dinâmicos. A modelagem começa com a representação do sistema por uma série infinita de integrais convolucionais, onde cada termo representa uma ordem de não linearidade e memória. A saída \( y(t) \) do sistema pode ser expressa pela equação \ref{eq:Volterra}:

\[
y(t) = h_0 + \sum_{n=1}^{\infty} \int_{-\infty}^{\infty} \cdots \int_{-\infty}^{\infty} h_n(\tau_1, \tau_2, \ldots, \tau_n) \prod_{i=1}^{n} x(t - \tau_i) \, d\tau_i
\label{eq:Volterra}
\]

onde \( h_n \) são os núcleos de Volterra, caracterizando a resposta do sistema para a \( n \)-ésima ordem de não linearidade, e \( x(t) \) é a entrada do sistema. Em muitas aplicações práticas, a série é truncada para incluir um número finito de termos devido à impraticabilidade de identificar todos os núcleos de uma série infinita \cite{Gonçalves2009}.
\subsection{Polinômio de memória}
O Polinômio de Memória (MP) é um modelo simples, utilizado na modelagem simplificada das séries de Volterra, considerando componentes unidimensionais. Este modelo é compacto, de baixo custo computacional e linear em seus parâmetros, sendo eficaz para amplificadores de potência (PAs) com pouco efeito de memória. Como o DPD e o pós-distorcedor apresentam características inversas às do PA, o mesmo modelo pode ser utilizado. A equação \ref{eq:mp} descreve o MP conforme ilustrado por \cite{Schuartz2017}:

\[
y(n) = \sum_{p=1}^{P} \sum_{m=0}^{M} h_{p,m} x(n - m) \left| x(n - m) \right|^{p-1}
\label{eq:mp}
\]

Para implementar esse modelo em hardware, é necessário paralelizar as operações aritméticas, visando altas taxas de operação. Nesse sentido, as FPGAs são uma alternativa viável para a implementação de circuitos pré-distorcedores.

\section{IMPLEMENTAÇÃO EM SOFTWARE} \label{sec:implsoft}
Nesta etapa, foi implementado o modelo DPD em software utilizando Python, com sinais de entrada e saída de um amplificador de potência classe AB baseado em HEMT GaN. O amplificador foi excitado por um sinal de 900 MHz modulado por WCDMA 3GPP, com largura de banda de aproximadamente 3,84 MHz. Os dados foram coletados usando um VSA Rohde \& Schwarz FSQ a 61,44 MHz de taxa de amostragem \cite{Bonfim2016}. 

A seguir, foi calculada a estimativa do sinal utilizando números de vírgula fixa e validada com o NMSE. Para isso, os dados foram divididos em conjuntos de extração e validação. A matriz de regressão foi calculada com os dados de extração para obter os coeficientes do polinômio de memória. O modelo foi validado com os dados de validação, resultando em um NMSE de -23,57 dB para um polinômio de 2° grau com uma amostra memorizada.

O algoritmo foi então ajustado para trabalhar com números em vírgula fixa, buscando a menor resolução possível e o menor NMSE simulado, o que exigiu a readequação dos resultados entre as multiplicações para manter a resolução inicial.

\section{IMPLEMENTAÇÃO EM FPGA}
Essa etapa envolveu a implementação do DPD em FPGA, o que exige a paralelização das operações aritméticas. Em cada ciclo de clock, três operações são realizadas simultaneamente: o sinal atual é elevado ao quadrado, armazenado em um registrador de deslocamento dentro de uma matriz de extração \footnote{matriz que contém todas as potências e amostras anteriores necessárias para o cálculo da saída}, o cálculo do produto de todos os elementos da matriz de extração e a soma dos produtos entre os sinais do mesmo instante e seus respectivos coeficientes. Esse processo se repete \( P \) vezes, correspondendo ao grau \( P \) do polinômio de memória. Como consequência, a saída do DPD estará incompleta durante os primeiros \( P \) ciclos de clock, pois, nesse intervalo, o cálculo depende de amostras de sinais anteriores que ainda não foram processadas, resultando em uma saída parcial.
A Figura \ref{fig:diagramaprocess} ilustra como esse processo está dividido entre cada ciclo de clock.

\begin{figure}[htbp!]
	\centering
	\captionsetup{justification=centering}
	\caption*{Fonte: Autor}
	\includegraphics[width=0.50\textwidth]{diagrama_process.png}
	\caption{Processo de cálculo da saída}
	\label{fig:diagramaprocess}
\end{figure}
Conforme exibido no diagrama, cada etapa do processo fornece os dados necessários para a próxima fase do cálculo com um atraso de um ciclo de clock. Contudo, é importante destacar que o processo completo demanda ciclos adicionais, uma vez que o sinal de saída só é registrado na borda de subida seguinte do clock, garantindo a sincronização adequada no fluxo de dados.


\section{RESULTADOS}
Conforme mencionado no capítulo \ref{chap:mati}, o desenvolvimento deste trabalho foi dividido em quatro etapas. A primeira etapa envolveu o estudo dos DPDs e dos métodos de modelagem associados. Na segunda etapa, essa modelagem foi implementada em software utilizando a linguagem Python. A terceira etapa consistiu na implementação do modelo de DPD selecionado em hardware, empregando a linguagem VHDL. Por fim, na quarta etapa, foi realizado a síntese lógica para o design do circuito integrado. Este capítulo apresenta os resultados obtidos ao longo do desenvolvimento do projeto.

\section{Modelagem do PA}

Para fazer a modelagem em software foi utilizada a linguagem de programação Python. Para isso, separou-se os dados citados na seção \ref{sec:implsoft} do capítulo \ref{chap:mati}, em dados de extração e dados de validação, os quais são utilizados para extração dos coeficientes do modelo do MP e validação do modelo encontrado, respectivamente. Para fazer a validação do modelo utilizou-se a métrica do NMSE, que consiste em calcular o erro médio quadrado do valor medido pelo VSA para o valor calculado pelo modelo. Portanto, quanto menor o NMSE mais fiel é o modelo do PA. Nesta etapa obteve-se um NMSE de -23.57 dB, para cálculos em vírgula flutuante, onde o resultado está presente no gráfico da figura \ref{fig:modelopafloat}.

\begin{figure}[ht!]
    \centering
    \captionsetup{justification=centering}
    \caption*{Fonte: Autor}
    \includegraphics[width=\figsize]{modeloPAfloat.png}
    \caption{Modelo do PA em vírgula flutuante}
    \label{fig:modelopafloat}
\end{figure}

\section{Apuração dos números de bits e resolução do sinal} 

Após concluída a modelagem matemática, foi feita a modelagem do DPD para então ser feito o levantamento da quantidade de bits necessários para a implementação do DPD em hardware minimizando os erros de quantização. 
Para isso foi necessário refazer a extração dos coeficientes, mas desta vez com os dados normalizados para valores de 0 a $2^{bits}$.  
O resultado desse levantamento está presente no gráfico na figura \ref{fig:bits}.

\begin{figure}[ht!]
    \centering
    \captionsetup{justification=centering}
    \caption*{Fonte: Autor}
    \includegraphics[width=\figsize]{bits.png}
    \caption{Gráfico Número de bits x NMSE}
    \label{fig:bits}
\end{figure}

Neste gráfico observa-se duas curvas, a curva em azul apresenta a quantidade total de bits total contando com os bits de overflow necessárias para as operações de multiplicação, enquanto a curva em vermelho representa a quantidade de bits de resolução do sinal. Analisando este gráfico observou-se que não existem ganhos significativos no erro a partir de 7 bits, portanto foi feita a modelagem do PA utilizando uma resolução de 8 bits o resultado alcançado está ilustrado pela figura \ref{fig:modelopa}.

\begin{figure}[ht!]
    \centering
    \captionsetup{justification=centering}
    \caption*{Fonte: Autor}
    \includegraphics[width=\figsize]{modeloPA.png}
    \caption{Modelo do PA em vírgula fixa}
    \label{fig:modelopa}
\end{figure}

\section{Modelagem do DPD}
A partir dos resultados obtidos foi possível fazer a modelagem do DPD, para isso foi feito o mesmo processo de modelagem do PA, porém para alcançar a característica de transferência inversa PA foi invertido a ordem dos dados de entrada e saída para extração dos coeficientes do DPD. O resultado desta modelagem está ilustrado pela figura \ref{fig:modelodpd} a seguir.

\begin{figure}[H]
    \centering
    \captionsetup{justification=centering}
    \caption*{Fonte: Autor}
    \includegraphics[width=\figsize]{modelodpd.png}
    \caption{Modelo do DPD em vírgula fixa}
    \label{fig:modelodpd}
\end{figure}

\section{Implementação em FPGA}
Em seguida foi feito o código em VHDL para a implementação em FPGA, nessa implementação cada operação aritmética é feita de maneira síncrona, e o fluxo dos cálculos desse processo esta sendo ilustrado pelo diagrama da figura \ref{fig:fluxocal} a seguir.

\begin{figure}[ht!]
	\centering
	\captionsetup{justification=centering}
	\caption*{Fonte: Autor}
	\includegraphics[width=\figsize]{fluxo_de_calculo.png}
	\caption{Fluxo de cálculo FPGA}
	\label{fig:fluxocal}
\end{figure}

No primeiro ciclo de clock é feito o registro do sinal de entrada, para em seguida ele ser elevado ao quadrado N graus do polinômio, para depois esse valor ser adicionado a outro buffer de matriz de extração para para todos os sinais de amostra e por fim ser multiplicado pelos seus respectivos e somado para o sinal de saída.

Essa descrição de hardware foi implementada FPGA Virtex5 XC5VLX50T, operando a uma frequência de 62,5 MHz, cujo os recursos lógicos estão sendo mostrados na tabela \ref{tab:recursos_fpga} a seguir.

\begin{table}[h!]
	\centering
	\begin{tabular}{|l|r|r|}
		\hline
		Recursos & Unidade & Percentual \\
		\hline
		Registradores & 150 & 1\% \\
		LUTs & 692 & 2\% \\
		DSP48E & 4 & 8\% \\
		\hline
	\end{tabular}
	\caption{Utilização dos recursos do FPGA no projeto analisado.}
	\label{tab:recursos_fpga}
\end{table}

Para fazer essa simulação foi utilizado um testbench com as mesmas entradas simuladas no python. Essa simulação foi feita no Xilinx ISE cujo o resultado está ilustrado na Figura \ref{fig:simise}.

\begin{figure}[ht!]
	\centering
	\captionsetup{justification=centering}
	\caption*{Fonte: Autor}
	\includegraphics[width=\figsize]{simulaçao ise.png}
	\caption{Simulação ISE}
	\label{fig:simise}
\end{figure}

O testbench utilizado nessa simulação gera um arquivo de texto com os sinais de saída para possibilitar a comparação dos sinais de saída em calculados em Python e no ISE. A figura \ref{fig:simfpga} ilustra o resultado dessa comparação.


\begin{figure}[ht!]
	\centering
	\captionsetup{justification=centering}
	\caption*{Fonte: Autor}
	\includegraphics[width=\figsize]{fpgasim.png}
	\caption{Simulação FPGA}
	\label{fig:simfpga}
\end{figure}


\section{Síntese lógica}
Por fim foi feito a síntese lógica do circuito e feita a simulação pós síntese lógica, cujo o resultado esta disponível na Figura \ref{fig:circuito_logico} e na tabela \ref{tab:recursos_sintese} 

\begin{figure}[ht!]
	\centering
	\captionsetup{justification=centering}
	\caption*{Fonte: Autor}
	\includegraphics[width=\figsize]{sim_pos_sin.png}
	\caption{Simulação pós síntese}
	\label{fig:circuito_logico}
\end{figure}

\begin{table}[h!]
	\centering
	\begin{tabular}{|l|r|r|}
		\hline
		Recursos & Unidade & Percentual \\
		\hline
		Registradores & 150 & 1\% \\
		LUTs & 692 & 2\% \\
		DSP48E & 4 & 8\% \\
		\hline
	\end{tabular}
	\caption{Utilização dos recursos de Células Lógicas.}
	\label{tab:recursos_sintese}
\end{table}

Esse circuito foi simulado utilizando o mesmo testbench utilizado na etapa anterior de design na FPGA, porém utilizando o NcLaunch do Cadence, ou seja, essa simulação gerou um arquivo de texto com os sinais de saída para depois ser feita a validação dos sinais de saida \ref{fig:simpost}.

\begin{figure}[ht!]
	\centering
	\captionsetup{justification=centering}
	\caption*{Fonte: Autor}
	\includegraphics[width=\figsize]{sim_pos_sin.png}
	\caption{Simulação pós síntese}
	\label{fig:simpost}
\end{figure}

1567 células lógica, com uma area total de 28116 $um^2$ e um consumo de energia de 1.6 mW

Esse circuito foi simulado atuando a uma taxa de operação de 20 MHz, ou seja, a síntese lógica apresentou um desempenho pior que o apresentado pela FPGA. 

\section{CONCLUSÃO}
A evolução dos sistemas de comunicação sem fio tem promovido a implementação de diversos serviços móveis, tornando essencial que esses sistemas operem com máxima eficiência. Nesse cenário, a implementação de um DPD em cascata com o PA surge como uma alternativa de baixo custo e interessante para melhorar o desempenho desses sistemas.
O objetivo deste trabalho de conclusão de curso foi implementar em hardware um DPD baseado no modelo de Polinômio de Memória. Para isso, o projeto foi dividido em quatro etapas: estudo do DPD e da modelagem matemática, modelagem do DPD em software, implementação do DPD em hardware e, finalmente, design do circuito integrado.
Sendo assim a primeira etapa de desenvolvimento do projeto foi a modelagem do PA em vírgula flutuante, utilizando o método do MP, para fazer essa modelagem utilizou-se um polinômio de $2^\circ$ grau com uma amostra de memória, para fazer a validação dessa modelagem utilizou-se a métrica do NMSE. Nesta etapa obteve-se um NMSE de -23,57 dB, a próxima etapa consistiu em otimizar a quantidade de células lógicas utilizadas no processo limitando o número de bits utilizados. Nesta etapa observou-se que a partir de 8 bits, não havia melhora expressiva no NMSE, assim, essa foi a resolução em bits utilizadas para a amostragem de sinais. Em seguida foi feita a modelagem do DPD em software o qual apresentou um comportamento inverso em relação ao do PA. Após isso foi feita a implementação do circuito em VHDL e validação em FPGA Virtex5 XC5VLX50T, que utilizou um total de 150 registradores, 692 LUTs e 4 unidades DSP48E, operando a uma frequência de 62,5 MHz. Em seguida seguiu-se para a etapa de síntese lógica a qual resultou em um circuito com 1567 células lógicas, com uma area total de 28116 $\mu m^2$ e um consumo de energia de 1.6 mW, atuando a uma taxa de operação de 33,34 MHz. Conclui-se, portanto, que o projeto alcançou os resultados esperados, com uma implementação eficaz do DPD em hardware, exceto pela taxa de operação da síntese lógica esperou-se uma melhor performance, o que não foi observado nas simulações.

\section*{REFERÊNCIAS}
\begingroup
\renewcommand{\section}[2]{}%

\begin{thebibliography}{}

		
		\bibitem{John2016} 
		Elton John. 
		\textit{Modelagem comportamental de amplificadores de potência de radiofrequência usando termos unidimensionais e bidimensionais de séries de Volterra}. 
		Universidade Federal do Paraná, 2016.
		
		\bibitem{Kenington2000} 
		Peter Kenington. 
		\textit{High Linearity RF Amplifier Design}. 
		Artech House, 2000.
		
		\bibitem{Cripps2006} 
		Steve Cripps. 
		\textit{RF Power Amplifiers for Wireless Communications}. 
		Artech House, 2006.
		
		\bibitem{Chavez2018} 
		Joel Huanca Chavez. 
		\textit{Estudo comparativo entre as arquiteturas de identificação de pré-distorcedores digitais através das aprendizagens direta e indireta}. 
		2018.
		
		\bibitem{Pedroni2010} 
		Volnei Pedroni. 
		\textit{Eletrônica Digital e VHDL}. 
		Elsevier, 2010.
		
		\bibitem{Goncalves2009} 
		Eduardo Gonçalves de Lima and Giovanni Ghione. 
		\textit{Behavioral modeling and digital base-band predistortion of RF power amplifiers}. 
		Politecnico di Torino, 2009.
		
		\bibitem{Schuartz2017} 
		Luis Schuartz and Eduardo Lima. 
		\textit{Polinômios com Memória de Complexidade Reduzida e sua Aplicação na Pré-distorção Digital de Amplificadores de Potência}. 
		Universidade Federal do Paraná, 2017.
		
		\bibitem{Bonfim2016} 
		Elton J. Bonfim and Eduardo G. De Lima. 
		``A Modified Two Dimensional Volterra-Based Series for the Low-Pass Equivalent Behavioral Modeling of RF Power Amplifiers". 
		\textit{IEEE Transactions on Microwave Theory and Techniques}, vol. 47, pp. 27-35, 2016.

	
\end{thebibliography}
\endgroup
\end{document}
