A evolução dos sistemas de comunicação sem fio tem promovido a implementação de diversos serviços móveis, tornando essencial que esses sistemas operem com máxima eficiência. Nesse cenário, a implementação de um DPD em cascata com o PA surge como uma alternativa de baixo custo e interessante para melhorar o desempenho desses sistemas.

Assim, validou-se a implementação de um código em linguagem de descrição de hardware capaz de processar em tempo real as características de transferência de um amplificador, reproduzindo suas não linearidades e efeitos de memória. Além disso, o código foi projetado para ocupar o mínimo possível de recursos lógicos do circuito digital que o sintetizará, visando reduzir o consumo de energia.

O desenvolvimento deste trabalho seguiu a seguinte sequência: inicialmente, foi realizada a modelagem do PA em vírgula flutuante, utilizando o método do Polinômio de Memória (MP) com um polinômio de $2^\circ$ grau e uma amostra de memória. A validação dessa modelagem foi feita com a métrica NMSE, obtendo-se um valor de -23,57 dB. Em seguida, foi realizada a otimização da quantidade de células lógicas no processo, limitando a resolução para 8 bits, após constatar que valores superiores não apresentavam melhoria significativa no NMSE. Essa resolução foi então utilizada para a amostragem dos sinais. Posteriormente, a modelagem do DPD foi feita em software, apresentando comportamento inverso ao do PA e atendendo aos requisitos do projeto. A etapa seguinte iniciou a implementação do modelo em hardware, e os próximos passos incluem a validação e otimização do circuito para alcançar a melhor performance possível.