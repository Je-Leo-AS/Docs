\chapter*{\normalsize ANEXO A - função que calcula matriz de extração em virgula flutuante}
\label{cod:mp}
\small % Altera o tamanho da fonte
\begin{lstlisting}[language = Python, ]
def mp(P, M, xn):
    L = xn.shape
    XX = np.zeros((L[0] - M, P * (M+1)), dtype=np.complex128)
    for l in range(M+1, L[0]):
        for p in range(1, P+1):
            for m in range(0, M+1):
                XX[l-M-1, ((p-1)*(M+1))+m] = (np.abs(xn[l-m])**(2*p-2)*(xn[l-m]))[0]
    return XX

XX_ext = mp(P, M, in_data_ext)
coefficients, _, _, _ = np.linalg.lstsq(XX_ext, out_data_ext[M:], rcond=None)
predicted_val = XX_val @ coefficients

\end{lstlisting}

\chapter*{\normalsize ANEXO B - Função que calcula matriz de confusão em virgula fixa}

readeq = lambda val, precision: np.floor(val / (2 ** precision))

\label{cod:mpint}
\small % Altera o tamanho da fonte
\begin{lstlisting}[language = Python, ]
    def mp_int(P, M, xn, bits):
    L = xn.shape
    XX = np.zeros((L[0] - M, P * (M+1)), dtype=np.complex128)
    for l in range(M+1, L[0]):
        for p in range(1, P+1):
            for m in range(0, M+1):
                    A = np.real(xn[l-m])[0]
                    B = np.imag(xn[l-m])[0]
                    modulo_power = 2**bits 
                    modulo_square = readeq(A ** 2 + B ** 2, bits)
                    for _ in range(1, p):
                       modulo_power = readeq(modulo_power * modulo_square, bits)
                    real_part = readeq(A * modulo_power,bits)
                    imag_part = readeq(B * modulo_power,bits)
                    XX[l-M-1, ((p-1)*(M+1))+m] = complex(real_part,imag_part)        
    return XX

\end{lstlisting}
