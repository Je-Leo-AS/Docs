A crescente demanda por eficiência espectral e energética nos sistemas de comunicação sem fio torna essencial o uso de técnicas capazes de mitigar os efeitos das não linearidades introduzidas por amplificadores de potência, sendo a modelagem comportamental desses dispositivos uma etapa fundamental nesse contexto. Este trabalho teve como objetivo estudar e validar a modelagem de um amplificador de potência por meio do modelo de Polinômio de Memória (MP), estabelecendo uma base consistente para aplicações futuras em técnicas de pré-distorção digital. Inicialmente, foi realizado um estudo teórico sobre os princípios do DPD e sobre a formulação matemática do modelo MP, evidenciando suas vantagens em termos de complexidade e capacidade de representação das não linearidades com efeitos de memória. Em seguida, a modelagem do amplificador foi desenvolvida em ambiente de software utilizando aritmética em vírgula flutuante, com extração dos coeficientes por meio de métodos de otimização numérica e validação baseada na métrica do Erro Quadrático Médio Normalizado (NMSE), cujos resultados demonstraram a eficácia do modelo MP clássico na caracterização do comportamento do PA e estabeleceram uma referência de desempenho para análises posteriores. A partir dessa base, foi proposta uma extensão do modelo MP na qual a ordem polinomial máxima passa a ser definida de forma independente para cada atraso de memória, introduzindo maior flexibilidade na modelagem e abrindo a possibilidade de reduzir a complexidade do modelo sem comprometer significativamente sua capacidade de representação, cuja validação quantitativa será realizada em etapas futuras deste trabalho. Adicionalmente, foram analisadas estratégias para redução da complexidade computacional do modelo, com destaque para o uso de tabelas de consulta (LUTs), explorando a separação entre envoltória e fase inerente ao modelo MP, embora essa abordagem tenha sido tratada apenas em nível conceitual nesta etapa do estudo. Dessa forma, o trabalho alcançou seu objetivo ao validar a modelagem comportamental do amplificador de potência em software e ao propor uma estrutura de modelagem mais flexível, além de delinear diretrizes claras para a utilização de LUTs como ferramenta para redução de complexidade, constituindo um passo inicial sólido para o desenvolvimento da versão final da dissertação e para estudos futuros na área de modelagem e linearização de amplificadores de potência.
