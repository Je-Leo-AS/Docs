\documentclass[
	12pt,				% tamanho da fonte
	openright,			% capátulos começam em página (insere páginavazia caso preciso)
	oneside,			% para impressão apenas em um lado do papel
	a4paper,			% tamanho do papel.
	brazil				% o último idioma é o principal do documento
	]{abntex2}
\usepackage{etoolbox}
\usepackage{lmodern}			% Usa a fonte Latin Modern		
\usepackage{longtable}

\usepackage[T1]{fontenc}		% Seleção de códigos de fonte.
\usepackage[utf8]{inputenc}		% Codificação do documento (conversão automática dos acentos)
%\usepackage{lastpage}			% Usado pela Ficha catalográfica
\usepackage{indentfirst}		% Indenta o primeiro parágrafo de cada seção.
\setlength{\parindent}{1.5cm}   % Espaçamento de 1,5cm do parágrafo
\usepackage{color}				% Controle das cores
\usepackage{graphicx}			% Inclusão de gráficos
\usepackage{microtype} 			% para melhorias de justificação
\usepackage{lipsum}				% para geração de dummy text
\usepackage[table,xcdraw]{xcolor}% Coluna colorida em tabelas
\usepackage{Capa}               % Capa e folha de rosto com modificações
\usepackage{float}              % Melhor posicionamento de figuras
\usepackage{gensymb}            % símbolo 
\usepackage[justification=justified,singlelinecheck=false]{caption}
%\usepackage{etoolbox}           % Configurações adicionais de macros
\usepackage{xparse}						
\usepackage{multirow}
\usepackage{textcomp}
\usepackage{amsmath}

\NewDocumentCommand\cc{+u{\cc}}{\ignorespaces}
\usepackage[justification=centering]{caption}
\usepackage{listings}
% \usepackage[backend=biber, style=abnt]{biblatex} 
%\addbibresource{Referencias.bib}
% \defbibheading{bibliography}{\chapter{Referências}} % Para mudar o título quando usa biblatex

% --------------------
% Dados do Documento
% --------------------

\lstset{
  basicstyle=\ttfamily,
  keywordstyle=\color{blue},
  commentstyle=\color{green},
  stringstyle=\color{red},
  numbers=left,
  numberstyle=\tiny\color{gray},
  stepnumber=1,
  numbersep=5pt,
  showspaces=false,
  showstringspaces=false,
  tabsize=2,
  breaklines=true,
  breakatwhitespace=false
}
\titulo{Modelagem Comportamental Híbrida de Amplificadores de Potência Usando Polinômios com Memória e Tabelas de Busca}
\autor{Leonardo de Andrade Santos}
\data{\the\year}
\instituicao{Universidade Federal do Paraná}
\local{Curitiba}
\orientador[Orientador:]{Eduardo Gonçalves de Lima}
\preambulo{Dissertação apresentada ao Programa de Pós-Graduação em Engenharia Elétrica, Área de
Concentração em Telecomunicações, Departamento de Engenharia Elétrica, Setor de Tecnologia, Universidade Federal do Paraná, como parte das exigências para obtenção do título de Mestre em Engenharia Elétrica.}


% ---------------------
% Configurações básicas
% --------------------

% informações do PDF
\makeatletter
\def\figsize{0.75\textwidth} % Define 80% da largura do texto
\hypersetup{
     	%pagebackref=true,
		pdftitle={\@title},
		pdfauthor={\@author},
    	pdfsubject={\imprimirpreambulo},
        pdfkeywords = {PARF}{VHDL}{DPD}{FPGA},
		colorlinks=true,       		% false: boxed links; true: colored links
    	linkcolor=black,          	% color of internal links
    	citecolor=black,      	    % color of links to bibliography
    	filecolor=magenta,      	% color of file links
		urlcolor=black,
		bookmarksdepth=4
}
\makeatother

\graphicspath{{Figuras/}}

% -------------------
% Início do documento
% -------------------

\begin{document}

\frenchspacing % Retira espaço extra obsoleto entre as frases.

% ----------------------------------------------------------
% ELEMENTOS PRÉ-TEXTUAIS
% ----------------------------------------------------------

% ----
% Capa
% ----
\imprimircapa
% ---

\newenvironment{NoTOC}{%
	\addtocontents{toc}{\protect\setcounter{tocdepth}{-1}}% Reduz a profundidade do TOC para ignorar conteúdo
}{%
	\addtocontents{toc}{\protect\setcounter{tocdepth}{2}}% Restaura a profundidade padrão do TOC
}

% --------------
% Folha de rosto
% --------------
\imprimirfolhaderosto
% ---

%\begin{dedicatoria}  % Opcional
%\vspace*{\fill}
% Dedico este trabalho aos meus pais .....
%\vspace*{\fill}
%\end{dedicatoria}

% --------------
% Agradecimentos  % Opcional
% --------------

% \begin{agradecimentos}

% Inclua seus agradecimentos aqui. Há vários exemplos na internet.
% \end{agradecimentos}

% \begin{epigrafe} % Opcional
% \vspace*{\fill}
% \begin{flushright}
% \textit{"Democracia é oportunizar a todos o mesmo ponto de partida. \\
%          Quanto ao ponto de chegada, depende de cada um".\\
%          (Fernando Sabino)}
% \end{flushright}
% \end{epigrafe}

% ------
% Resumo
% ------
\newpage
\setlength{\absparsep}{18pt}   % ajusta o espaçamento dos parágrafos do resumo
\setlength{\abstitleskip}{1cm} % adiciona mais um cm após o 'titulo' do Resumo para ficar com 2cm,

% -------------------------------
% Lista de abreviaturas e siglas - Opcional
% -------------------------------
\begin{resumo}
	
	A evolução dos sistemas de comunicação sem fio possibilitou o surgimento de diversas aplicações móveis e wireless, como desenvolvimento web e Internet das Coisas (IoT). Nesse contexto, a melhoria da eficiência energética é altamente desejável, tanto em dispositivos móveis, que buscam maior autonomia de bateria, quanto em estações rádio-base, que visam reduzir perdas associadas à dissipação de calor. Entretanto, o aumento da eficiência energética geralmente implica a redução da linearidade dos amplificadores de potência (PAs) utilizados nos transmissores de rádio. Essa limitação é particularmente crítica em sistemas modernos de comunicação, nos quais a largura de banda disponível é restrita e o uso de modulações complexas, que exploram variações simultâneas de amplitude e fase, é essencial para alcançar elevadas taxas de transmissão. Modulações sensíveis à amplitude tornam-se especialmente suscetíveis às não linearidades do PA, resultando em degradação do desempenho e aumento dos erros de transmissão.

Uma solução amplamente empregada para conciliar eficiência energética e linearidade é a utilização da pré-distorção digital (Digital Predistortion – DPD) em cascata com o PA. A eficácia dessa técnica depende diretamente da disponibilidade de modelos matemáticos capazes de representar com precisão o comportamento não linear e com memória do amplificador. Neste trabalho, inicialmente investiga-se uma variação do modelo Memory Polynomial (MP), na qual a ordem polinomial passa a depender do atraso de memória, permitindo maior flexibilidade na modelagem e redução da complexidade computacional sem prejuízo significativo de desempenho. A partir dessa análise, avalia-se uma abordagem híbrida que combina polinômios e tabelas de consulta (Look-Up Tables – LUT), priorizando o uso de LUT nos termos associados às amostras atuais, que concentram maior complexidade, e mantendo a implementação polinomial para os termos de memória. Por fim, são comparadas as abordagens puramente polinomial, puramente baseada em LUT e a abordagem híbrida, considerando métricas de precisão e complexidade computacional.
 
\textbf{Palavras-chave}: DPD, Polinômio de memória, LUTs.

	.
\end{resumo}

% \renewcommand{\resumoname}{Abstract} % Redefine o título temporariamente
% \begin{resumo}
%     The evolution of wireless communication systems has led to the implementation of various mobile and wireless applications, such as web development and IoT applications, among others. In this context, improving energy efficiency becomes a desirable alternative both for mobile devices aiming to enhance battery autonomy and for base radio stations seeking to reduce heat loss waste. However, better energy efficiency implies lower linearity in the signal amplification systems present in radio signal transmitters. This is significant because the bandwidth allocated for mobile applications is limited, meaning that achieving higher transmission rates requires alternating modulation strategies for both the phase and amplitude of the carrier wave. These two conditions are conflicting since AM modulation is sensitive to linearity, and the more linear a system is, the fewer transmission errors occur. Thus, an alternative to overcoming this challenge—implementing a system that is both energy-efficient and linear—is the use of a Digital Predistortion (DPD) system in cascade with a Power Amplifier (PA). Therefore, the goal of this undergraduate thesis is the design of a dedicated integrated circuit for a DPD. To achieve this goal, the project was divided into four stages: studying and modeling DPDs, DPD modeling in software, DPD implementation on FPGA, and finally, designing the DPD integrated circuit. For DPD modeling, the NMSE (Normalized Mean Square Error) metric was used, where a lower NMSE indicates a model that is more faithful to reality. In the PA modeling stage, an NMSE of -23,57 dB was achieved. Next, the number of bits required to perform these calculations while minimizing NMSE was determined. It was found that with only 8 bits of signal resolution, it was already possible to achieve an NMSE close to the floating-point value. After this analysis, the circuit was implemented in VHDL and validated on an FPGA Virtex5 XC5VLX50T, using a total of 150 registers, 692 LUTs, and 4 DSP48E units, operating at a frequency of 61,5 MHz. Subsequently, the logical synthesis stage was carried out, resulting in a circuit with 1,567 logic cells, a total area of 28,116 $um^2$, and power consumption of 1,6 mW, operating at a frequency of 33,34 MHz.

\textbf{Palavras-chave}: VHDL, FPGA, DPD 
% \end{resumo}
% \renewcommand{\resumoname}{Resumo}
\begin{siglas}
	\item[DPD] Pré-Distorcedor Digital
	\item[FPGA] Field-Programmable Gate Array (Matriz de Portas Programáveis em Campo)
	\item[PA] Amplificador de Potência
	\item[RF] Radio Frequency (Rádio Frequência)
	\item[PARF] Amplificador de Potência de Rádio Frequência
	\item[HDL] Hardware Description Language (Linguagem de Descrição de Hardware)
	\item[VHSIC] Very High-Speed Integrated Circuit (Circuito Integrado de Velocidade Muito Elevada)
	\item[VHDL] VHSIC Hardware Description Language
	\item[LUT] Look-Up Table
	\item[SOP] Sum of Products (Soma de Produtos)
	\item[LAB] Logic Array Block
	\item[ALM] Adaptive Logic Module
	\item[LE] Logic Element
	\item[HEMT] High Electron Mobility Transistor (Transistor de Efeito de Campo de Heterojunção)
	\item[VSA] Vector Signal Analyzer (Analisador de Sinal Vetorial)
	\item[NMSE] Normalized Mean Squared Error (Erro Médio Quadrado Normalizado)
\end{siglas}





	
\listoffigures*
\cleardoublepage{\tiny }
\listoftables*
\cleardoublepage
% -------
% Sumario
% -------
\pdfbookmark[0]{\contentsname}{toc}
\tableofcontents*
\cleardoublepage
% ---

\makepagestyle{abntheadings}
\makeevenhead{abntheadings}{\ABNTEXfontereduzida\thepage}{}{}
\makeoddhead{abntheadings}{}{}{\ABNTEXfontereduzida\thepage}
\makeheadrule{abntheadings}{\textwidth}{0in}

% ----------------------------------------------------------
% ELEMENTOS TEXTUAIS
% ----------------------------------------------------------

\textual

%\setcounter{page}{7}

\chapter{Introdução} 
\label{chap:introducao}
A evolução dos sistemas de comunicação móveis, impulsionada pela crescente demanda por comunicações mais rápidas e eficientes, tem levado à implementação de uma variedade de serviços, incluindo aplicações multimídia, desenvolvimento web e aplicações IoT \cite{John2016}. No entanto, essa evolução também trouxe desafios significativos, como a necessidade de melhorar a eficiência energética, tanto para dispositivos móveis, visando aumentar a autonomia da bateria, quanto para estações de rádio base, visando reduzir o consumo de energia devido às perdas de calor. Para atender a essas demandas, estratégias de modulação que alteram tanto a fase quanto a amplitude de ondas portadoras em radiofrequência se tornaram essenciais \cite{Kenington2000}. Além disso, a modulação na amplitude requer linearidade na transmissão para evitar erros e interferências na comunicação entre usuários vizinhos \cite{Cripps2006}. Essa complexa tarefa recai sobre o projetista do amplificador de potência de radiofrequência (PA), que enfrenta o desafio de desenvolver um hardware eficiente em termos energéticos e linear ao mesmo tempo, uma vez que esses dois objetivos podem entrar em conflito \cite{Chavez2018}. Uma solução para contornar esse desafio é a implementação de um pré-distorcedor digital (DPD) em banda base, que visa compensar a distorção causada pelo PA \cite{Cripps2006}. O DPD é conectado em cascata ao PA e requer um modelo de alta precisão e baixa complexidade computacional para representar as características de transferência direta e inversa do amplificador.

Existem diversas abordagens para modelar o PA. Neste projeto, adota-se uma modelagem híbrida, combinando polinômios e tabelas de consulta (LUT). Especificamente, para amostras atuais — em que a ordem polinomial necessária é alta —, prioriza-se o uso de LUT; para amostras passadas — com ordens polinomiais menores —, mantém-se o emprego de polinômios. Será realizada uma comparação entre as abordagens puramente polinomial, puramente baseada em LUT e a abordagem híbrida, avaliando-se complexidade computacional e precisão de modelagem.

Neste contexto, este projeto foi planejado com os objetivos detalhados a seguir.

\section{Objetivo Geral}
Investigar, em software Python, estruturas alternativas do modelo \textit{Memory Polynomial} (MP) para a modelagem matemática de amplificadores de potência (PA), com ênfase em abordagens que utilizam ordens polinomiais dependentes do atraso, avaliando o compromisso entre complexidade computacional e precisão do modelo.

\section{Objetivos Específicos}
Para alcançar o objetivo geral, este trabalho foi desenvolvido com base nos seguintes objetivos específicos:
\begin{enumerate}
    \item Implementar, em software Python, o modelo \textit{Memory Polynomial} tradicional para a modelagem comportamental de amplificadores de potência;
    
    \item Desenvolver e implementar uma variação do modelo MP com ordens polinomiais dependentes do atraso, permitindo truncamentos polinomiais distintos para cada ramo de memória;
    
    \item Avaliar o desempenho dos modelos propostos por meio de métricas de erro, com ênfase no \textit{Normalized Mean Square Error} (NMSE);
    
    \item Analisar a relação entre a complexidade estrutural dos modelos, expressa pelo número de coeficientes, e a precisão obtida, por meio de uma abordagem de otimização multiobjetivo baseada na fronteira de Pareto;
    
    \item Identificar configurações de modelos MP que apresentem melhor compromisso entre desempenho e complexidade, visando futuras implementações eficientes.
\end{enumerate}


\chapter{Revisão de Literatura} 
\label{chap:revi}
A evolução dos sistemas de comunicações sem fio tem impulsionado o desenvolvimento de diversas aplicações móveis. Nesse contexto, a eficiência energética emerge como uma característica essencial, beneficiando tanto a autonomia de baterias em dispositivos móveis quanto a redução de perdas em estações rádio base, onde energia é dissipada principalmente na forma de calor.

O sistema de comunicação pode ser dividido em três subsistemas principais: transmissor, receptor e meio de propagação, conforme descrito por \cite{Schuartz2017}. Este trabalho concentra-se exclusivamente no subsistema transmissor, ilustrado na Figura \ref{fig:sistemadetrasmissao}, que inclui diversos componentes responsáveis pela geração e amplificação do sinal. Dentre esses, o PARF destaca-se como o elemento de maior consumo energético, pois converte energia CC da fonte em energia RF irradiada pela antena. Assim, a eficiência global do transmissor depende diretamente do desempenho do PARF.

\begin{figure}[htbp!]
    \centering
    \captionsetup{justification=centering}
    \caption*{Fonte: \cite{Schuartz2017}}
    \includegraphics[width=0.5\textwidth]{sistematrasmissorpng.png}
    \caption{Sistema de transmissão simplificado}
    \label{fig:sistemadetrasmissao}
\end{figure}

\section{Sistema transmissor}

O componente central do PARF é o transistor, responsável pela amplificação da potência do sinal de entrada proveniente de estágios anteriores. Nesse processo, energia CC das fontes de alimentação é convertida em energia CA. Para maximizar a potência de saída, o PARF deve apresentar alta eficiência, definida como a relação entre a potência entregue à carga ($P_{\text{out}}$) e a potência consumida da fonte CC ($P_{\text{cc}}$):

\begin{equation}
    \eta = \frac{P_{\text{out}}}{P_{\text{cc}}} \times 100\%
    \label{eq:rendimento}
\end{equation}

Devido a imperfeições nos componentes e no ambiente operacional, a eficiência ideal de 100\% nunca é alcançada. Além do transistor, o circuito do PARF inclui redes de casamento de impedância (entrada e saída), que otimizam a transferência de potência, e um circuito de polarização CC, que fornece as condições operacionais adequadas ao transistor. Esses elementos, compostos por capacitores e indutores, introduzem efeitos dinâmicos (efeitos de memória). A Figura \ref{fig:circuitoparf} apresenta um circuito simplificado de PARF.

\begin{figure}[htbp!]
    \centering
    \captionsetup{justification=centering}
    \caption*{Fonte: \cite{Luiza2016}}
    \includegraphics[width=0.5\linewidth]{Figuras/circuito parf.png}
    \caption{Circuito simplificado de um PARF}
    \label{fig:circuitoparf}
\end{figure}

Tipicamente, a eficiência do PA aumenta com a potência de saída, sendo máxima próximo à saturação. Potência não convertida em sinal útil é dissipada como calor, elevando custos de projeto e comprometendo a confiabilidade devido a temperaturas excessivas. Assim, maximizar a eficiência do PARF é crucial para redes de telecomunicações.

Outra característica fundamental é a linearidade, avaliada pela curva de transferência (relação entre potência de saída e entrada, em dBm), como exemplificado na Figura \ref{fig:saidaparf}. Nessa curva, observa-se uma região linear inicial, seguida de compressão de ganho (ponto de compressão de 1 dB), onde o ganho reduz 1 dB em relação ao regime de baixa potência. Próximo à saturação, o ganho diminui progressivamente, degradando a eficiência.

\begin{figure}[htbp!]
    \centering
    \captionsetup{justification=centering}
    \caption*{Fonte: \cite{Chavez2018}}
    \includegraphics[width=0.5\textwidth]{curvasaidaparf.png}
    \caption{Curva de transferência de um PARF}
    \label{fig:saidaparf}
\end{figure}

A largura de banda disponível para sistemas de comunicação sem fio é um recurso naturalmente limitado, o que torna essencial sua utilização de maneira eficiente. À medida que a demanda por maiores taxas de transmissão de dados cresce, surge a necessidade de empregar técnicas de modulação capazes de transmitir mais informação dentro da mesma faixa espectral. Nesse contexto, conforme apresentado por \cite{Kenington2000}, as maiores taxas de transmissão somente podem ser alcançadas por meio de esquemas de modulação que exploram simultaneamente variações na fase e na amplitude da onda portadora em radiofrequência.

Entretanto, o uso de modulações que envolvem variações de amplitude impõe requisitos mais rigorosos sobre a linearidade dos sistemas de transmissão. Segundo \cite{Kenington2000}, a falta de linearidade pode resultar em distorções do sinal transmitido, ocasionando erros na comunicação e o surgimento de interferências indesejadas em canais adjacentes, prejudicando usuários vizinhos. Dessa forma, garantir um comportamento linear ao longo da cadeia de transmissão torna-se um aspecto crítico para a qualidade e confiabilidade do sistema de comunicação.

Nesse cenário, o projeto do PARF assume um papel central, uma vez que esse componente é responsável por fornecer potência suficiente ao sinal modulado antes de sua transmissão pela antena. O principal desafio enfrentado pelo projetista do PARF consiste em conciliar dois requisitos conflitantes: a obtenção de alta eficiência energética e a manutenção de uma boa linearidade. Conforme discutido por \cite{Cripps2006}, amplificadores de potência tendem a apresentar maior eficiência quando operam próximos à região de saturação. Contudo, nessa região de operação, o dispositivo passa a apresentar um comportamento fortemente não linear.

Assim, embora a operação próxima à saturação seja desejável do ponto de vista do consumo de energia, ela compromete a linearidade do amplificador, resultando em distorções no sinal transmitido. Esse compromisso entre eficiência e linearidade é inerente ao funcionamento do PARF e constitui um dos principais desafios no projeto de sistemas modernos de comunicação sem fio, conforme ilustrado pela Figura \ref{fig:saidaparf}.


\subsection{Comportamento Passa Banda do PA}

Nos sistemas modernos de telecomunicações, a transmissão de dados é realizada por meio de sinais em radiofrequência (RF), cujas frequências centrais situam-se tipicamente na ordem dos GHz. Esses sinais são modulados por uma envoltória complexa, responsável por carregar a informação, cuja largura de banda encontra-se na faixa dos MHz. Como a largura de banda do sinal modulado é significativamente menor do que a frequência da portadora, tais sinais são classificados como sinais passa banda, conforme descrito por \cite{Luiza2016}.

Uma forma conveniente de analisar sinais passa banda consiste em representá-los por meio de sua forma equivalente em base-baixa. Essa representação separa a portadora de alta frequência da envoltória complexa, permitindo uma análise mais intuitiva dos efeitos introduzidos pelo sistema de transmissão, especialmente no que se refere às variações de amplitude e fase. Essa abordagem é amplamente utilizada na modelagem comportamental de amplificadores de potência, pois facilita a identificação e a caracterização das distorções causadas pelas não linearidades do circuito do PA.

O amplificador de potência em radiofrequência (PA) desempenha um papel fundamental na cadeia de transmissão, sendo responsável por fornecer potência suficiente ao sinal antes de sua irradiação pela antena. Entretanto, os dispositivos ativos que compõem o PA apresentam comportamento inerentemente não linear, especialmente quando operam próximos à região de saturação. Como consequência, essas não linearidades afetam diretamente sinais passa banda, que possuem múltiplas componentes espectrais concentradas em torno da frequência central.

\begin{figure}[htbp!]
    \centering
    \captionsetup{justification=centering}
    \caption*{Fonte: \cite{Luiza2016}}
    \includegraphics[width=1.0\linewidth]{comportamento passa banda.png}
    \caption{Exemplo de distorção no espectro de Frequência do PARF}
    \label{fig:comportamentopassabanda}
\end{figure}


A Figura \ref{fig:comportamentopassabanda} ilustra o comportamento típico de um PA passa banda, evidenciando os sinais de entrada e saída tanto no domínio do tempo quanto no domínio da frequência. Observa-se que, enquanto o espectro do sinal de entrada está confinado à banda desejada, o sinal de saída apresenta um espalhamento espectral. Esse fenômeno ocorre devido à geração de produtos de intermodulação provocados pelas não linearidades do amplificador, resultando no surgimento de componentes espectrais fora da banda original do sinal.

Esse espalhamento espectral é particularmente indesejável em sistemas de comunicação sem fio, pois pode causar interferência em canais adjacentes, degradando o desempenho de usuários vizinhos e violando requisitos regulatórios de emissão espectral. Além disso, a presença de distorções no sinal transmitido compromete a qualidade da comunicação e reduz a eficiência espectral do sistema. Dessa forma, a compreensão do comportamento passa banda do PA é essencial para o desenvolvimento de técnicas de linearização, como a pré-distorção digital, que visam mitigar os efeitos das não linearidades e garantir a integridade do sinal transmitido, conforme discutido em \cite{Luiza2016}.

\section{Linearização de Amplificadores de Potência}

Conforme discutido na seção anterior, as não linearidades inerentes aos amplificadores de potência em radiofrequência (PARFs) causam distorções significativas em sinais passa banda, resultando em espalhamento espectral e interferência em canais adjacentes. Esse efeito torna-se especialmente crítico nos sistemas modernos de comunicação sem fio, nos quais a largura de banda disponível é limitada e há uma demanda crescente por maiores taxas de transmissão de dados. Para atender a esses requisitos, são amplamente empregadas modulações digitais complexas que variam simultaneamente a amplitude e a fase do sinal, como QAM e OFDM \cite{Kenington2000}.

Entretanto, tais esquemas de modulação impõem elevados requisitos de linearidade ao sistema de transmissão, uma vez que qualquer não linearidade introduzida pelo PARF afeta diretamente a envoltória do sinal, comprometendo sua integridade e degradando a qualidade da comunicação. Por outro lado, a operação do amplificador em regiões estritamente lineares geralmente ocorre longe da saturação, o que implica baixa eficiência energética. Esse comportamento evidencia o compromisso fundamental entre eficiência e linearidade nos PARFs, amplamente discutido na literatura \cite{Cripps2006}. A baixa eficiência resulta em maior dissipação térmica, reduzindo a autonomia de dispositivos móveis alimentados por bateria e elevando os custos operacionais em estações rádio base.

Diante desse cenário, diversas técnicas de linearização têm sido desenvolvidas com o objetivo de mitigar os efeitos das não linearidades do PARF, permitindo sua operação em regiões mais eficientes sem comprometer a qualidade do sinal transmitido. Dentre essas técnicas, a pré-distorção digital (Digital Predistortion – DPD) destaca-se pelo seu favorável compromisso entre desempenho e custo de implementação \cite{Kenington2000}. A técnica de DPD consiste em aplicar, em banda base, uma distorção controlada ao sinal de entrada, de forma que a sua característica de transferência seja aproximadamente inversa à do PARF.

Quando o sinal pré-distorcido é aplicado ao amplificador, as não linearidades do PARF compensam a distorção introduzida pelo DPD, resultando em um comportamento global aproximadamente linear do sistema em cascata. Esse conceito é ilustrado na Figura \ref{fig:cascatadpd}, que apresenta o esquema de um pré-distorcedor digital operando em conjunto com o PARF para o qual foi projetado. Para que essa compensação seja eficaz, torna-se necessário que o DPD seja capaz de representar com precisão não apenas o comportamento não linear estático do amplificador, mas também os seus efeitos de memória.

\begin{figure}[htbp!]
    \centering
    \captionsetup{justification=centering}
    \caption*{Fonte: \cite{Chavez2018}}
    \includegraphics[width=0.5\textwidth]{DPDcascata.png}
    \caption{Esquema de pré-distorcedor digital em cascata com PARF}
    \label{fig:cascatadpd}
\end{figure}

Nesse contexto, a modelagem comportamental constitui uma etapa fundamental no desenvolvimento de técnicas de pré-distorção digital. O diagrama de blocos apresentado na Figura \ref{fig:diagramamodelagem} exemplifica essa abordagem, na qual um modelo matemático é submetido ao mesmo sinal de entrada aplicado ao amplificador de potência, representado por $x(t)$. A saída simulada do modelo, $y_{sim}(t)$, é então comparada com a saída real do PARF, $y_{real}(t)$.

Os coeficientes do modelo são ajustados a partir do erro entre $y_{real}(t)$ e $y_{sim}(t)$, por meio de algoritmos de otimização cujo objetivo é minimizar esse erro. Quando o erro mínimo é alcançado, o modelo é considerado otimizado e capaz de reproduzir adequadamente o comportamento do amplificador de potência. Nessa condição, o modelo pode ser empregado na implementação do pré-distorcedor digital, possibilitando a linearização do PARF e a mitigação do espalhamento espectral, conforme descrito em \cite{Luiza2016}.

\begin{figure}[htbp!]
    \centering
    \caption*{Fonte: \cite{Luiza2016}}
    \includegraphics[width=0.5\linewidth]{diagrama simulação.png}
    \caption{Diagrama de Modelagem Comportamental}
    \label{fig:diagramamodelagem}
\end{figure}

\section{Modelagens Matemáticas}

Conforme discutido na seção anterior, a técnica de pré-distorção digital (DPD) depende diretamente da capacidade de representar com precisão o comportamento não linear do amplificador de potência em radiofrequência (PARF). Para que a linearização seja eficaz, o pré-distorcedor deve reproduzir, de forma inversa, as características do amplificador, compensando tanto as não linearidades estáticas quanto os efeitos dinâmicos associados à memória do dispositivo. Dessa forma, torna-se indispensável o uso de modelos matemáticos capazes de descrever adequadamente o comportamento do PA sob diferentes condições de operação.

Nos sistemas modernos de comunicação sem fio, a limitação de largura de banda disponível leva à adoção de esquemas de modulação com elevada variação de envoltória, caracterizados por altos valores de Peak-to-Average Power Ratio (PAPR). Esses sinais impõem requisitos rigorosos ao PARF, que deve operar de forma eficiente do ponto de vista energético sem comprometer a linearidade. Para atender a essas exigências, as técnicas de linearização demandam modelos computacionais precisos do comportamento do amplificador, conforme discutido em \cite{John2016}.

De maneira geral, as abordagens de modelagem de amplificadores de potência podem ser classificadas em duas categorias principais: modelagem física e modelagem comportamental. A modelagem física baseia-se no conhecimento detalhado da topologia do circuito, dos dispositivos semicondutores e dos componentes passivos que constituem o amplificador. Embora essa abordagem possa oferecer elevada precisão, sua aplicação é limitada pela alta complexidade computacional e pela dificuldade de obtenção de todos os parâmetros físicos necessários. Em contrapartida, a modelagem comportamental, também conhecida como empírica, fundamenta-se exclusivamente na observação da relação entre os sinais de entrada e saída do sistema, sem a necessidade de informações detalhadas sobre a estrutura interna do circuito. Essa característica torna os modelos comportamentais particularmente atrativos para aplicações de simulação e linearização, devido à sua menor complexidade computacional.

No contexto da pré-distorção digital, a modelagem comportamental é amplamente preferida, uma vez que o objetivo principal é reproduzir o comportamento inverso observável do PARF. Além das não linearidades estáticas, os efeitos de memória — isto é, a dependência da saída atual em relação a amostras passadas do sinal de entrada — desempenham um papel relevante no comportamento do amplificador. Dessa forma, os sistemas podem ser classificados como sem memória ou com memória, sendo esta última categoria a mais representativa para amplificadores operando com sinais de larga banda.

Dentre os diversos modelos comportamentais propostos na literatura, destacam-se os modelos polinomiais com memória e as redes neurais artificiais. Embora as redes neurais apresentem elevada capacidade de aproximação, sua implementação pode acarretar maior complexidade computacional. Neste trabalho, opta-se por modelos baseados em simplificações da série de Volterra, priorizando estruturas polinomiais devido ao seu compromisso favorável entre capacidade de modelagem e custo computacional.

\subsection{Séries de Volterra}

A série de Volterra constitui uma extensão da série de Taylor para a representação de sistemas não lineares dinâmicos com memória, sendo amplamente utilizada na modelagem de amplificadores de potência \cite{GonalvesdeLima2009}. Por meio dessa abordagem, a saída do sistema é expressa como uma combinação de integrais múltiplas envolvendo o sinal de entrada e núcleos que caracterizam o comportamento do sistema em diferentes ordens de não linearidade.

Matematicamente, a saída $y(t)$ de um sistema descrito pela série de Volterra pode ser expressa como:

\begin{equation}
    y(t) = h_0 + \sum_{n=1}^{\infty} \int_{-\infty}^{\infty} \cdots \int_{-\infty}^{\infty} h_n(\tau_1, \ldots, \tau_n) \prod_{i=1}^{n} x(t - \tau_i) \, d\tau_i
    \label{eq:Volterra}
\end{equation}

em que $h_n(\tau_1, \ldots, \tau_n)$ representa o núcleo de Volterra de ordem $n$, responsável por descrever os efeitos não lineares e de memória do sistema. Embora a série de Volterra forneça uma descrição bastante geral e precisa do comportamento de sistemas não lineares com memória, sua aplicação prática é limitada pela elevada complexidade computacional. Por esse motivo, na maioria das aplicações, a série é truncada para ordens finitas, restringindo o número de termos considerados.

\subsection{Polinômio com Memória}

O modelo de polinômio com memória (\textit{Memory Polynomial} – MP) surge como uma simplificação da série de Volterra, obtida pela consideração apenas dos termos diagonais, isto é, daqueles que compartilham o mesmo atraso temporal \cite{Schuartz2017}. Essa simplificação reduz significativamente a complexidade do modelo, ao mesmo tempo em que preserva a capacidade de representar não linearidades e efeitos de memória de forma satisfatória para uma ampla classe de amplificadores de potência.

O modelo MP pode ser descrito matematicamente por:

\begin{equation}
    y(n) = \sum_{p=1}^{P} \sum_{m=0}^{M} h_{p,m} \, x(n - m) \left| x(n - m) \right|^{p-1}
    \label{eq:mp}
\end{equation}

em que $P$ representa a ordem de não linearidade do modelo, $M$ corresponde à profundidade de memória considerada, e $h_{p,m}$ são os coeficientes do modelo. Uma das principais vantagens do modelo de polinômio com memória é sua linearidade nos parâmetros, o que facilita a estimação dos coeficientes por meio de técnicas de otimização convencionais. Além disso, esse modelo apresenta boa eficácia na representação de amplificadores de potência com efeitos de memória moderados, sendo amplamente empregado em aplicações de pré-distorção digital.

Observa-se que, na formulação tradicional do modelo MP, a ordem polinomial máxima $P$ é adotada de forma uniforme para todos os termos de memória, independentemente do atraso considerado. Essa restrição, embora simplifique a estrutura do modelo, não é imposta pela formulação original da série de Volterra e pode resultar em um aumento desnecessário da complexidade computacional, especialmente para termos associados a atrasos mais elevados, cujos efeitos não lineares tendem a ser menos pronunciados. Dessa forma, a adoção de ordens polinomiais diferenciadas em função do atraso de memória surge como uma alternativa potencial para reduzir a complexidade do modelo, mantendo sua capacidade de representação.

Do ponto de vista de implementação em hardware, especialmente em sistemas que operam com altas taxas de amostragem, torna-se fundamental explorar arquiteturas eficientes que permitam a paralelização das operações aritméticas. Nesse contexto, avalia-se a substituição de operações de multiplicação por tabelas de consulta (\textit{Look-Up Tables} – LUTs), visando reduzir a complexidade computacional e viabilizar a implementação prática do modelo em sistemas embarcados e plataformas digitais.

\section{Uso de Look-Up Tables em Pré-Distorção Digital}

A implementação prática de técnicas de pré-distorção digital (DPD) em sistemas de comunicação sem fio de alta taxa impõe restrições significativas de complexidade computacional, especialmente quando se considera a operação em tempo real e com sinais de larga banda. Nesse contexto, estratégias que reduzam o custo computacional da linearização, sem comprometer de forma significativa o desempenho, têm sido amplamente investigadas na literatura. Dentre essas estratégias, destaca-se o uso de tabelas de consulta, conhecidas como LUTs, como alternativa eficiente para a implementação de pré-distorcedores digitais.

As LUTs consistem em estruturas de memória que armazenam previamente valores calculados de uma determinada função, permitindo que operações matemáticas complexas sejam substituídas por simples acessos à memória. Em aplicações de DPD, as LUTs podem ser utilizadas para mapear diretamente a envoltória do sinal de entrada para valores de correção de amplitude e fase, implementando, de forma discreta, a função inversa do PARF. Essa abordagem elimina a necessidade de operações aritméticas intensivas, como multiplicações e potências, reduzindo significativamente a latência e o consumo de recursos computacionais.

O trabalho apresentado por \cite{Kwan2012} demonstra a viabilidade da implementação de um pré-distorcedor digital baseado em LUTs em plataformas do tipo \textit{Field Programmable Gate Array} (FPGA), utilizando sinais Long Term Evolution (LTE) com largura de banda de até 60 MHz. Nesse estudo, os autores evidenciam que a abordagem baseada em LUTs é capaz de linearizar amplificadores de potência operando com sinais de banda larga, atendendo aos requisitos de emissão espectral impostos por padrões modernos de comunicação, ao mesmo tempo em que mantém uma arquitetura de implementação simples e eficiente.

Uma das principais vantagens do uso de LUTs em DPD é a adequação natural desse tipo de estrutura a arquiteturas paralelas, como as encontradas em FPGAs. O acesso à memória pode ser realizado de forma altamente paralelizável e determinística, favorecendo a implementação em tempo real mesmo em cenários de elevada taxa de amostragem. Além disso, a granularidade da LUT pode ser ajustada conforme os requisitos de precisão e os recursos disponíveis, permitindo um compromisso controlável entre desempenho de linearização e consumo de memória.

Entretanto, o uso de LUTs também apresenta desafios, como a necessidade de um processo eficiente de preenchimento e atualização da tabela, bem como a escolha adequada do número de entradas para evitar quantização excessiva da função de pré-distorção. Conforme discutido em \cite{Kwan2012}, técnicas de interpolação e estratégias de endereçamento podem ser empregadas para mitigar esses efeitos e melhorar a precisão do modelo sem aumento significativo da complexidade.

Diante desse panorama, o uso de LUTs surge como uma alternativa promissora para a implementação de pré-distorção digital em amplificadores de potência, especialmente em aplicações que demandam elevada largura de banda e operação em tempo real. No contexto deste trabalho, as LUTs são consideradas como uma possível estratégia para reduzir a complexidade computacional dos modelos polinomiais com memória, viabilizando sua implementação em hardware e ampliando a aplicabilidade prática das técnicas de linearização estudadas.


\chapter{Material e Métodos} 
\label{chap:mati}
Como mencionado anteriormente, este trabalho tem como objetivo desenvolver o design de um circuito integrado para um DPD, partindo de um modelo previamente validado tanto em software quanto em hardware, especificamente em FPGA. O projeto foi dividido em quatro etapas principais:

\begin{itemize}
\item Estudo sobre PA e modelagem matemática;
\item Implementação em software do PA e do DPD;
\item Implementação em do DPD em FPGA;
\item Design e validação do circuito implementado com a tecnologia de 8HP 130nm.
\end{itemize}


\section{Estudo sobre PA e modelagem matemática}
A etapa consistiu no estudo de modelagens de Amplificadores de potência para posteriormente fazer a modelagem do DPDs, conforme apresentado no Capítulo \ref{chap:revi}, na qual foi feito todo o levantamento sobre os tipos de modelagem dos DPDs. O objetivo deste estudo é entender as diferentes abordagens de modelagem, avaliar seus desempenhos e identificar as mais adequadas para a aplicação em amplificadores de potência.

\section{Implementação em software} \label{sec:implsoft}

Nesta etapa, foi realizada a implementação do modelo DPD em software, utilizando a linguagem de programação Python. Esta linguagem amigável e amplamente difundida na comunidade acadêmica.

Para essa modelagem, foram coletados sinais de entrada e saída de um amplificador de potência classe AB, que utiliza um HEMT fabricado com tecnologia GaN. O amplificador foi excitado por um sinal portador de frequência de 900 MHz, modulado por um sinal de envelope WCDMA 3GPP com aproximadamente 3,84 MHz de largura de banda. Os dados de entrada e saída do amplificador de potência foram medidos usando um VSA Rohde \& Schwarz FSQ com uma taxa de amostragem de 61,44 MHz, conforme disponível em \cite{Bonfim2016}.

Em seguida, realizou-se o cálculo da estimativa do sinal utilizando números com vírgula fixa. Para verificar a precisão dessa estimativa em relação ao sinal original, calculou-se o NMSE. Para essa validação, os dados foram inicialmente divididos em conjuntos de extração e validação. A matriz de extração foi calculada com os dados de extração, utilizando o código disponível no anexo \ref{cod:mp}. Esse cálculo é essencial para a extração dos coeficientes do polinômio de memória. Após a extração dos coeficientes, calculou-se o modelo do PA, que foi então validado com os dados de validação. O NMSE obtido para um polinômio de $2^\circ$ grau com uma amostra memorizada foi de -23,57 dB.

Em seguida, o algoritmo foi ajustado para operar com números em vírgula fixa e o número total de bits foi reajustado para atingir a menor resolução possível, buscando o menor NMSE simulado, conforme ilustrado pelo anexo \ref{cod:mpint}. Por ser tratar de um cálculo em vírgula fixa, fez-se necessário uma readequação do resultado obtido entre cada multiplicação de forma a manter a resolução inicial.

\section{Implementação em FPGA}
Essa etapa consiste na implementação do DPD em FPGA. Para isso, é necessário realizar paralelizações nas operações aritméticas. A Figura \ref{fig:diagramaprocess} ilustra como esse processo está dividido entre cada ciclo de clock. Aqui está o trecho reescrito de forma mais clara: A cada ciclo, duas operações são realizadas em paralelo: o sinal atual é elevado ao quadrado e registrado, enquanto ocorre o somatório do produto entre os sinais do mesmo instante de tempo e seus respectivos coeficientes. Esse processo ocorre P vezes para os \( P \) graus do polinômio de memória. Portanto, a saída do DPD é incompleta para os primeiros \( P \) períodos de clock, pois, nesses primeiros ciclos, realiza-se o cálculo com base em entradas de sinais anteriores que ainda não ocorreram, resultando em uma saída incompleta.

\begin{figure}[ht!]
  \centering
  \captionsetup{justification=centering}
  \caption*{Fonte: Autor}
  \includegraphics[width=0.80\textwidth]{diagrama_process.png}
  \caption{Processo de cálculo da saída}
  \label{fig:diagramaprocess}
\end{figure}

\section{Design e validação}
Finalmente, na última etapa, realiza-se o processo de concepção da síntese lógica do circuito do DPD  na tecnologia BiCMOS 130 nm 8HP, utilizando as ferramentas específicas de desing de circuito integrado. 



\chapter{Resultados e Discussão} 
\label{chap:resul}
Conforme mencionado no capítulo \ref{chap:mati}, o desenvolvimento deste trabalho foi dividido em quatro etapas. A primeira etapa envolveu o estudo dos DPDs e dos métodos de modelagem associados. Na segunda etapa, essa modelagem foi implementada em software utilizando a linguagem Python. A terceira etapa consistiu na implementação do modelo de DPD selecionado em hardware, empregando a linguagem VHDL. Por fim, na quarta etapa, foi realizado a síntese lógica para o design do circuito integrado. Este capítulo apresenta os resultados obtidos ao longo do desenvolvimento do projeto.

\section{Modelagem do PA}

Para fazer a modelagem em software foi utilizada a linguagem de programação Python. Para isso, separou-se os dados citados na seção \ref{sec:implsoft} do capítulo \ref{chap:mati}, em dados de extração e dados de validação, os quais são utilizados para extração dos coeficientes do modelo do MP e validação do modelo encontrado, respectivamente. Para fazer a validação do modelo utilizou-se a métrica do NMSE, que consiste em calcular o erro médio quadrado do valor medido pelo VSA para o valor calculado pelo modelo. Portanto, quanto menor o NMSE mais fiel é o modelo do PA. Nesta etapa obteve-se um NMSE de -23.57 dB, para cálculos em vírgula flutuante, onde o resultado está presente no gráfico da figura \ref{fig:modelopafloat}.

\begin{figure}[ht!]
    \centering
    \captionsetup{justification=centering}
    \caption*{Fonte: Autor}
    \includegraphics[width=\figsize]{modeloPAfloat.png}
    \caption{Modelo do PA em vírgula flutuante}
    \label{fig:modelopafloat}
\end{figure}

\section{Apuração dos números de bits e resolução do sinal} 

Após concluída a modelagem matemática, foi feita a modelagem do DPD para então ser feito o levantamento da quantidade de bits necessários para a implementação do DPD em hardware minimizando os erros de quantização. 
Para isso foi necessário refazer a extração dos coeficientes, mas desta vez com os dados normalizados para valores de 0 a $2^{bits}$.  
O resultado desse levantamento está presente no gráfico na figura \ref{fig:bits}.

\begin{figure}[ht!]
    \centering
    \captionsetup{justification=centering}
    \caption*{Fonte: Autor}
    \includegraphics[width=\figsize]{bits.png}
    \caption{Gráfico Número de bits x NMSE}
    \label{fig:bits}
\end{figure}

Neste gráfico observa-se duas curvas, a curva em azul apresenta a quantidade total de bits total contando com os bits de overflow necessárias para as operações de multiplicação, enquanto a curva em vermelho representa a quantidade de bits de resolução do sinal. Analisando este gráfico observou-se que não existem ganhos significativos no erro a partir de 7 bits, portanto foi feita a modelagem do PA utilizando uma resolução de 8 bits o resultado alcançado está ilustrado pela figura \ref{fig:modelopa}.

\begin{figure}[ht!]
    \centering
    \captionsetup{justification=centering}
    \caption*{Fonte: Autor}
    \includegraphics[width=\figsize]{modeloPA.png}
    \caption{Modelo do PA em vírgula fixa}
    \label{fig:modelopa}
\end{figure}

\section{Modelagem do DPD}
A partir dos resultados obtidos foi possível fazer a modelagem do DPD, para isso foi feito o mesmo processo de modelagem do PA, porém para alcançar a característica de transferência inversa PA foi invertido a ordem dos dados de entrada e saída para extração dos coeficientes do DPD. O resultado desta modelagem está ilustrado pela figura \ref{fig:modelodpd} a seguir.

\begin{figure}[H]
    \centering
    \captionsetup{justification=centering}
    \caption*{Fonte: Autor}
    \includegraphics[width=\figsize]{modelodpd.png}
    \caption{Modelo do DPD em vírgula fixa}
    \label{fig:modelodpd}
\end{figure}

\section{Implementação em FPGA}
Em seguida foi feito o código em VHDL para a implementação em FPGA, nessa implementação cada operação aritmética é feita de maneira síncrona, e o fluxo dos cálculos desse processo esta sendo ilustrado pelo diagrama da figura \ref{fig:fluxocal} a seguir.

\begin{figure}[ht!]
	\centering
	\captionsetup{justification=centering}
	\caption*{Fonte: Autor}
	\includegraphics[width=\figsize]{fluxo_de_calculo.png}
	\caption{Fluxo de cálculo FPGA}
	\label{fig:fluxocal}
\end{figure}

No primeiro ciclo de clock é feito o registro do sinal de entrada, para em seguida ele ser elevado ao quadrado N graus do polinômio, para depois esse valor ser adicionado a outro buffer de matriz de extração para para todos os sinais de amostra e por fim ser multiplicado pelos seus respectivos e somado para o sinal de saída.

Essa descrição de hardware foi implementada FPGA Virtex5 XC5VLX50T, operando a uma frequência de 62,5 MHz, cujo os recursos lógicos estão sendo mostrados na tabela \ref{tab:recursos_fpga} a seguir.

\begin{table}[h!]
	\centering
	\begin{tabular}{|l|r|r|}
		\hline
		Recursos & Unidade & Percentual \\
		\hline
		Registradores & 150 & 1\% \\
		LUTs & 692 & 2\% \\
		DSP48E & 4 & 8\% \\
		\hline
	\end{tabular}
	\caption{Utilização dos recursos do FPGA no projeto analisado.}
	\label{tab:recursos_fpga}
\end{table}

Para fazer essa simulação foi utilizado um testbench com as mesmas entradas simuladas no python. Essa simulação foi feita no Xilinx ISE cujo o resultado está ilustrado na Figura \ref{fig:simise}.

\begin{figure}[ht!]
	\centering
	\captionsetup{justification=centering}
	\caption*{Fonte: Autor}
	\includegraphics[width=\figsize]{simulaçao ise.png}
	\caption{Simulação ISE}
	\label{fig:simise}
\end{figure}

O testbench utilizado nessa simulação gera um arquivo de texto com os sinais de saída para possibilitar a comparação dos sinais de saída em calculados em Python e no ISE. A figura \ref{fig:simfpga} ilustra o resultado dessa comparação.


\begin{figure}[ht!]
	\centering
	\captionsetup{justification=centering}
	\caption*{Fonte: Autor}
	\includegraphics[width=\figsize]{fpgasim.png}
	\caption{Simulação FPGA}
	\label{fig:simfpga}
\end{figure}


\section{Síntese lógica}
Por fim foi feito a síntese lógica do circuito e feita a simulação pós síntese lógica, cujo o resultado esta disponível na Figura \ref{fig:circuito_logico} e na tabela \ref{tab:recursos_sintese} 

\begin{figure}[ht!]
	\centering
	\captionsetup{justification=centering}
	\caption*{Fonte: Autor}
	\includegraphics[width=\figsize]{sim_pos_sin.png}
	\caption{Simulação pós síntese}
	\label{fig:circuito_logico}
\end{figure}

\begin{table}[h!]
	\centering
	\begin{tabular}{|l|r|r|}
		\hline
		Recursos & Unidade & Percentual \\
		\hline
		Registradores & 150 & 1\% \\
		LUTs & 692 & 2\% \\
		DSP48E & 4 & 8\% \\
		\hline
	\end{tabular}
	\caption{Utilização dos recursos de Células Lógicas.}
	\label{tab:recursos_sintese}
\end{table}

Esse circuito foi simulado utilizando o mesmo testbench utilizado na etapa anterior de design na FPGA, porém utilizando o NcLaunch do Cadence, ou seja, essa simulação gerou um arquivo de texto com os sinais de saída para depois ser feita a validação dos sinais de saida \ref{fig:simpost}.

\begin{figure}[ht!]
	\centering
	\captionsetup{justification=centering}
	\caption*{Fonte: Autor}
	\includegraphics[width=\figsize]{sim_pos_sin.png}
	\caption{Simulação pós síntese}
	\label{fig:simpost}
\end{figure}

1567 células lógica, com uma area total de 28116 $um^2$ e um consumo de energia de 1.6 mW

Esse circuito foi simulado atuando a uma taxa de operação de 20 MHz, ou seja, a síntese lógica apresentou um desempenho pior que o apresentado pela FPGA. 

\chapter{Conclusão} 
\label{chap:conc}
A evolução dos sistemas de comunicação sem fio tem promovido a implementação de diversos serviços móveis, tornando essencial que esses sistemas operem com máxima eficiência. Nesse cenário, a implementação de um DPD em cascata com o PA surge como uma alternativa de baixo custo e interessante para melhorar o desempenho desses sistemas.
O objetivo deste trabalho de conclusão de curso foi implementar em hardware um DPD baseado no modelo de Polinômio de Memória. Para isso, o projeto foi dividido em quatro etapas: estudo do DPD e da modelagem matemática, modelagem do DPD em software, implementação do DPD em hardware e, finalmente, design do circuito integrado.
Sendo assim a primeira etapa de desenvolvimento do projeto foi a modelagem do PA em vírgula flutuante, utilizando o método do MP, para fazer essa modelagem utilizou-se um polinômio de $2^\circ$ grau com uma amostra de memória, para fazer a validação dessa modelagem utilizou-se a métrica do NMSE. Nesta etapa obteve-se um NMSE de -23,57 dB, a próxima etapa consistiu em otimizar a quantidade de células lógicas utilizadas no processo limitando o número de bits utilizados. Nesta etapa observou-se que a partir de 8 bits, não havia melhora expressiva no NMSE, assim, essa foi a resolução em bits utilizadas para a amostragem de sinais. Em seguida foi feita a modelagem do DPD em software o qual apresentou um comportamento inverso em relação ao do PA. Após isso foi feita a implementação do circuito em VHDL e validação em FPGA Virtex5 XC5VLX50T, que utilizou um total de 150 registradores, 692 LUTs e 4 unidades DSP48E, operando a uma frequência de 62,5 MHz. Em seguida seguiu-se para a etapa de síntese lógica a qual resultou em um circuito com 1567 células lógicas, com uma area total de 28116 $\mu m^2$ e um consumo de energia de 1.6 mW, atuando a uma taxa de operação de 33,34 MHz. Conclui-se, portanto, que o projeto alcançou os resultados esperados, com uma implementação eficaz do DPD em hardware, exceto pela taxa de operação da síntese lógica esperou-se uma melhor performance, o que não foi observado nas simulações.

% ----------------------------------------------------------
% Referências bibliográficas
% ----------------------------------------------------------

%\setlength{\afterchapskip}{\baselineskip}

\chapter{REFERÊNCIAS}
\begin{NoTOC} 
\begingroup
\renewcommand{\chapter}[2]{}%

\bibliographystyle{plain} 
\bibliography{Referencias}

\end{NoTOC}
\endgroup

% ----------------------------------------------------------
% ELEMENTOS PÓS-TEXTUAIS
% ----------------------------------------------------------
\postextual
% ----------------------------------------------------------

% ----------------------------------------------------------
% ANEXOS
% ----------------------------------------------------------

\begin{apendicesenv}

\partapendices

\chapter*{\normalsize APÊNDICE A - Digite o cabeçalho do apêndice}

Apêndice: texto ou documento elaborado pelo autor, a fim de complementar sua argumentação, sem prejuízo da unidade nuclear do trabalho.

\chapter*{\normalsize APÊNDICE B - Digite o cabeçalho do apêndice}

\end{apendicesenv}

%\begin{anexosenv}


%\addcontentsline{toc}{chapter}{\hspace{2.105cm}}
%\renewcommand{\ABNTEXchapterfontsize}{\ABNTEXsectionfont}

%\chapter{Código fonte}
O código-fonte desenvolvido para os cálculos deste trabalho está disponível em um repositório público na plataforma GitHub.  

Link para o repositório:  
\url{https://github.com/Je-Leo-AS/DPD}  

O repositório contém:
\begin{itemize}
	\item Código principal para cálculos.
	\item Scripts auxiliares para validação.
	\item Documentação técnica. 
\end{itemize}     

Para facilitar o acesso, utilize o QR Code abaixo:  

\begin{figure}[H]
	\centering
	\includegraphics[width=0.3\textwidth]{qrcode_diretorio_codigo.png}
	\caption{QR Code para o repositório do código-fonte.}
\end{figure}



%\end{anexosenv}


\end{document}