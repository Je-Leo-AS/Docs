A evolução dos sistemas de comunicação sem fio possibilitou o surgimento de diversas aplicações móveis e wireless, como desenvolvimento web e Internet das Coisas (IoT). Nesse contexto, a melhoria da eficiência energética é altamente desejável, tanto em dispositivos móveis, que buscam maior autonomia de bateria, quanto em estações rádio-base, que visam reduzir perdas associadas à dissipação de calor. Entretanto, o aumento da eficiência energética geralmente implica a redução da linearidade dos amplificadores de potência (PAs) utilizados nos transmissores de rádio. Essa limitação é particularmente crítica em sistemas modernos de comunicação, nos quais a largura de banda disponível é restrita e o uso de modulações complexas, que exploram variações simultâneas de amplitude e fase, é essencial para alcançar elevadas taxas de transmissão. Modulações sensíveis à amplitude tornam-se especialmente suscetíveis às não linearidades do PA, resultando em degradação do desempenho e aumento dos erros de transmissão.

Uma solução amplamente empregada para conciliar eficiência energética e linearidade é a utilização da pré-distorção digital (Digital Predistortion – DPD) em cascata com o PA. A eficácia dessa técnica depende diretamente da disponibilidade de modelos matemáticos capazes de representar com precisão o comportamento não linear e com memória do amplificador. Neste trabalho, inicialmente investiga-se uma variação do modelo Memory Polynomial (MP), na qual a ordem polinomial passa a depender do atraso de memória, permitindo maior flexibilidade na modelagem e redução da complexidade computacional sem prejuízo significativo de desempenho. A partir dessa análise, avalia-se uma abordagem híbrida que combina polinômios e tabelas de consulta (Look-Up Tables – LUT), priorizando o uso de LUT nos termos associados às amostras atuais, que concentram maior complexidade, e mantendo a implementação polinomial para os termos de memória. Por fim, são comparadas as abordagens puramente polinomial, puramente baseada em LUT e a abordagem híbrida, considerando métricas de precisão e complexidade computacional.
 
\textbf{Palavras-chave}: DPD, Polinômio de memória, LUTs.
